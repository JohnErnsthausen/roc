\usepackage{listings}
\usepackage{xcolor}


\definecolor{myred}{rgb}{0.50, 0. 0.}
\definecolor{keywordscolor}{rgb}{0,0,0.50}
\newcommand\basicfont{\tt}


% Solarized colour scheme for listings
\definecolor{solarized@base03}{HTML}{002B36}
\definecolor{solarized@base02}{HTML}{073642}
\definecolor{solarized@base01}{HTML}{586e75}
\definecolor{solarized@base00}{HTML}{657b83}
\definecolor{solarized@base0}{HTML}{839496}
\definecolor{solarized@base1}{HTML}{93a1a1}
\definecolor{solarized@base2}{HTML}{EEE8D5}
\definecolor{solarized@base3}{HTML}{FDF6E3}
\definecolor{solarized@yellow}{HTML}{B58900}
\definecolor{solarized@orange}{HTML}{CB4B16}
\definecolor{solarized@red}{HTML}{DC322F}
\definecolor{solarized@magenta}{HTML}{D33682}
\definecolor{solarized@violet}{HTML}{6C71C4}
\definecolor{solarized@blue}{HTML}{268BD2}
\definecolor{solarized@cyan}{HTML}{2AA198}
\definecolor{solarized@green}{HTML}{859900}

% Define C++ syntax highlighting colour scheme
\lstset{language=C++,
        basicstyle=\ttfamily,
       % numbers=left,
        numberstyle=\footnotesize,
        tabsize=2,
        breaklines=true,
        columns=flexible,
        breaklines=true,
  showstringspaces=false,
  emptylines=0,
  keepspaces=true,
        escapeinside={@}{@},
        numberstyle=\tiny\color{solarized@base01},
        keywordstyle=\color{myred},
        stringstyle=\color{solarized@cyan}\ttfamily,
        identifierstyle=\color{solarized@blue},
        commentstyle=\color{solarized@base01},
        emphstyle=\color{solarized@red},
        frame=single,
        rulecolor=\color{solarized@base2},
        rulesepcolor=\color{solarized@base2},
        showstringspaces=false,
        rangeprefix=/\*,
       rangesuffix=\*/,
       includerangemarker=false,
       escapeinside={/*!}{!*/},
       texcl=true,
  %    literate={:=}{{$\gets$}}2 {<=}{{$\leq$}}1 {>=}{{$\geq$}}1 {<>}{{$\neq$}}1 {*}{$\cdot$}1
}


%\lstset{style=cpp}

%in math or text mode \@xyz@ puts xyz in typewriter font
\def\@#1@{{\basicfont\color{solarized@blue} #1}}
%for typesetting strings
\nc\sr[1]{\@{"#1"}@} 


\nc{\inpcode}[2]{
\vspace{\bigskipamount}
\lstinputlisting[linerange=#1-#1,label={lst:#1}]{#2}{\index{\lstinline[language=bash]{#2}!\lstinline{#1}}}
}


\nc\lsref[1]{\hyperref[lst:#1]{\lstinline{#1}}}

%parameters
\newlist{parameters}{description}{1} 
\setlist[parameters]{
    itemsep = -1ex,
    itemindent=-2ex,
    topsep=1ex,
    format=\parameterslabel,
}
\newcommand\parameterslabel[2][l]{\eqmakebox[listlabel@\EnumitemId][#1]{#2}}

\def\begp{\begin{parameters}}
\def\endp{\end{parameters}}

%template  parameter
\def\tps #1  {\item [\@<#1>@]}  
%in parameter
\def\ins #1 {\item [\@#1@] }        
%out parameter
\def\outs #1 {\item  [$\rightarrow$ {\@#1@}]}  
\nc\return{\paragraph{Returns}}
\nc\example{\paragraph{Example.}}
\nc\result{\paragraph{Result.}}
\nc\note{\paragraph{Note.}}





