\documentclass[12pt, titlepage]{article}

\usepackage{amsmath, mathtools}
\usepackage{amsfonts}
\usepackage{amssymb}
\usepackage{booktabs}
\usepackage{tabularx}
\usepackage{xspace}

\usepackage{graphicx}
\usepackage{colortbl}
\usepackage{xr}
\usepackage{longtable}
\usepackage{xfrac}
\usepackage{float}
\usepackage{siunitx}
\usepackage{caption}
\usepackage{geometry}
\usepackage{pdflscape}
\usepackage{afterpage}

\usepackage{fullpage}
\usepackage[round]{natbib}
\usepackage{multirow}
%\usepackage{refcheck}
\usepackage{lipsum}

\usepackage[section]{placeins}
\usepackage{array}

%% Comments

\usepackage{color}

\newif\ifcomments\commentsfalse

\ifcomments
\newcommand{\authornote}[3]{\textcolor{#1}{[#3 ---#2]}}
\newcommand{\todo}[1]{\textcolor{red}{[TODO: #1]}}
\else
\newcommand{\authornote}[3]{}
\newcommand{\todo}[1]{}
\fi

\newcommand{\wss}[1]{\authornote{blue}{SS}{#1}} 
\newcommand{\plt}[1]{\authornote{magenta}{TPLT}{#1}} %For explanation of the template
\newcommand{\an}[1]{\authornote{cyan}{Author}{#1}}
\newcommand{\jme}[1]{\authornote{cyan}{JME}{#1}}

% Common Parts

\usepackage{tabularx}
\usepackage{booktabs}
\usepackage{ifthen}
\usepackage{amsmath,amssymb,xspace}

%PUT YOUR PROGRAM NAME HERE %Every program should have a name
\newcommand{\progname}[1]
{%
  \ifthenelse{\equal{#1}{n}}{{\normalsize ROC\xspace}}{}%
  \ifthenelse{\equal{#1}{L}}{{\Large ROC\xspace}}{}%
  \ifthenelse{\equal{#1}{f}}{{\footnotesize ROC\xspace}}{}%
}
% Abbreviations
\newcommand{\ode}{{\footnotesize ODE}\xspace}
\newcommand{\dae}{{\footnotesize DAE}\xspace}
\newcommand{\ivp}{{\footnotesize IVP}\xspace}
\newcommand{\fadbad}{{\footnotesize FADBAD++}\xspace}
\newcommand{\plainc}{{\footnotesize C}\xspace}
\newcommand{\cpp}{{\footnotesize C++}\xspace}
\newcommand{\adolc}{{\footnotesize ADOL-C}\xspace}
\newcommand{\rdcon}{{\footnotesize DRDCV}\xspace}
\newcommand{\fortran}{{\footnotesize FORTRAN 77}\xspace}
\newcommand{\daets}{{\footnotesize DAETS}\xspace}
\newcommand{\matlab}{{\footnotesize MATLAB}\xspace}
\newcommand{\mathematica}{{\footnotesize MATHEMATICA}\xspace}
\newcommand{\maple}{{\footnotesize MAPLE}\xspace}
% User commands
\newcommand{\Quote}[1]{``{#1}''}
\newcommand{\Ni}{\noindent}
% Environment
\newcommand{\EQ}[1]{\begin{align} {#1} \end{align}}
% Mathematics
\newcommand{\parn}[1]{( {#1} )}
\newcommand{\parbg}[1]{\left(  {#1} \right)}
\newcommand{\pbg}[1]{\bigl(  {#1} \bigr)}
\newcommand{\Setbg}[1]{\bigl\{ {#1} \bigr\}}
\def\Rz{\mathbb{R}}
\def\Cz{\mathbb{C}}
\newcommand{\nliminf}[1]{\liminf\limits_{{#1} \rightarrow \infty}}
\newcommand{\nlimsup}[1]{\limsup\limits_{{#1} \rightarrow \infty}}
% Mathematics: ODE
\newcommand{\iode}{\protect{\makebox{$[t_0,\tend]$}}\xspace}
\newcommand{\lode}{\protect{\makebox{$[t_n,t_{n+1}]$}}\xspace}
\newcommand{\tend}{t_\text{end}}
\newcommand{\tc}[2]{(#1)_{#2}}
\newcommand{\Tp}{T}
\newcommand{\xn}{x_{n}}
\newcommand{\tn}{t_{n}}
% Reference
\renewcommand{\eqref}[1]{(\ref{eq:#1})}
\newcommand{\rrf}[2]{(\ref{eq:#1}--\ref{eq:#2})}
\newcommand{\chref}[1]{\ref{ch:#1}}
\newcommand{\sscref}[1]{\ref{ssc:#1}}
\newcommand{\scref}[1]{Section~\ref{sc:#1}}
\newcommand{\exref}[1]{\ref{ex:#1}}
\newcommand{\rmref}[1]{\ref{rm:#1}}
\newcommand{\apref}[1]{\ref{ap:#1}}
%\newcommand{\tbref}[1]{\ref{tb:#1}}
\newcommand{\dfref}[1]{\ref{df:#1}}
\newcommand{\leref}[1]{\ref{le:#1}}
\newcommand{\fgref}[1]{\ref{fg:#1}}
\newcommand{\coref}[1]{\ref{co:#1}}
\newcommand{\thref}[1]{\ref{th:#1}}
\newcommand{\agref}[1]{\ref{ag:#1}}
\newcommand{\asref}[1]{\ref{as:#1}}
\newcommand{\EQref}[1]{Equation~(\ref{eq:#1})}
\newcommand{\CHref}[1]{Chapter~\ref{ch:#1}}
\newcommand{\SSCref}[1]{Subsection~\ref{ssc:#1}}
\newcommand{\SCref}[1]{Section~\ref{sc:#1}}
\newcommand{\EXref}[1]{Example~\ref{ex:#1}}
\newcommand{\RMref}[1]{Remark~\ref{rm:#1}}
\newcommand{\APref}[1]{Appendix~\ref{ap:#1}}
\newcommand{\TBref}[1]{Table~\ref{tb:#1}}
\newcommand{\DFref}[1]{Definition~\ref{df:#1}}
\newcommand{\LEref}[1]{Lemma~\ref{le:#1}}
\newcommand{\FGref}[1]{Figure~\ref{fg:#1}}
\newcommand{\COref}[1]{Corollary~\ref{co:#1}}
\newcommand{\THref}[1]{Theorem~\ref{th:#1}}
\newcommand{\AGref}[1]{Algorithm~\ref{ag:#1}}
\newcommand{\ASref}[1]{Assumption~\ref{as:#1}}



\makeatletter
\newcommand*{\addFileDependency}[1]{% argument=file name and extension
  \typeout{(#1)}
  \@addtofilelist{#1}
  \IfFileExists{#1}{}{\typeout{No file #1.}}
}
\makeatother

\newcommand*{\myexternaldocument}[1]{%
    \externaldocument{#1}%
    \addFileDependency{#1.tex}%
    \addFileDependency{#1.aux}%
}

\externaldocument{../SRS/SRS}

\newcommand{\rref}[1]{(R\ref{#1})}
\newcommand{\nfrref}[1]{NFR\ref{#1}}

\begin{document}

\title{Module Interface Specification for \progname{L}: Software estimating the radius of convergence of
a power series}
\author{John M Ernsthausen}
\date{\today}

\maketitle

\pagenumbering{roman}

\section{Revision History}

\begin{tabularx}{\textwidth}{p{4cm}p{2cm}X}
\toprule {\bf Date} & {\bf Version} & {\bf Notes}\\
\midrule
  22 November 2020 & 1.0 & First submission\\
  24 December 2020 & 2.0 & Second submission\\
\bottomrule
\end{tabularx}

\newpage

\section{Symbols, Abbreviations and Acronyms}

Symbols, abbreviations, and acronyms applicable to \progname{f} are enumerated
in Section 1 of the Software Requirements Document (SRS) \citep{SRS}.

\newpage

\tableofcontents

\newpage

\pagenumbering{arabic}

\section{Introduction}

% Move to MIS
%
%
%  \subsection{Summary}
%  
%  In \cite{SRS}, we detailed the algorithm implementing top-line analysis.
%  
%  \paragraph{Module findrc} \label{findrc}
%  We know the tolerance $\tol$, the scale $h$, the integer $N$, and the sequence
%  $\Setbg{c_0, c_1, \ldots, c_{N}}$ by assumption, as in \SCref{introduction}.
%  We extract the last $15$ terms from the sequence.
%  With $k = i+N-14$, obtain the best linear fit $y(k) = m k + b$ in the $2$-norm to the points
%  \EQ
%  {
%    \label{eq:approximation-points}
%    \Setbg{(N-14, \log_{10} | c_{N-14} |), (N-13, \log_{10} |c_{N-13}|), \ldots, (N, \log_{10} |c_{N}|)},
%  }
%  that is, find $m$ and $b$ such that $\sum_{i=0}^{14} |\log_{10} |c_{i+N-14}| - y(i+N-14)|^2$ is minimized.
%  The approximation problem \eqref{approximation-points} is the well known
%  {\it linear least-squares problem} \cite{GoVL89}.
%  The model parameter $m$ will be negative whenever the series \eqref{power-series} converges and
%  positive whenever it diverges.The radius of convergence $R_c = h/10^m$.
%  
%  \paragraph{Module findmu}
%  Start with the series resulting from integration of the given series three times and
%  fit the coefficients with Module {\bf findrc}. If that graph is linear, meaning the
%  minimizer has norm less than \tol, then the slope is accepted and the order of the
%  singularity is 3. If the graph opens upward, then the series is differentiated term-wise
%  to reduce the second derivative of the graph, and a new top-line fit is computed with
%  Module {\bf findrc}. This process is repeated, reducing $\mu$ by 1 each time, until the
%  graph opens downward or until seven term-wise differentiations have been tested.
%  If seven term-wise differentiations have been tested and each result in turn proves
%  unsatisfactory, then the final estimate for $R_c$ is reduced by 10 percent for a
%  conservative estimate for $R_c$ and $\mu=-4$ is returned.
%  
%  We implement Module {\bf findrc} and Module {\bf findrc} in \progname{f}.
%  This document provides plan for verification and validation \progname{f} while implementing
%  these modules in \cpp.
%  
%  Our objective it to implement Module {\bf findrc} and Module {\bf findrc} via
%  {\it test driving} in \cpp. The implementation must satisfy the requirements
%  enumerated in \cite{SRS}.



The following document details the Module Interface Specifications for the implemented
modules in library software \progname{f} estimating the radius of convergence of a power series.
It is intended to ease navigation through the program for design and maintenance purposes.

Complementary documents include the System Requirement Specifications and Module Guide.
The full documentation and implementation can be found at the project's
\href{https://github.com/JohnErnsthausen/roc}{GitHub repository}.

\section{Notation}

\wss{You should describe your notation.  You can use what is below as
  a starting point.}

The structure of the MIS for modules comes from \citet{HoffmanAndStrooper1995},
with the addition that template modules have been adapted from
\cite{GhezziEtAl2003}.  The mathematical notation comes from Chapter 3 of
\citet{HoffmanAndStrooper1995}.  For instance, the symbol := is used for a
multiple assignment statement and conditional rules follow the form $(c_1
\Rightarrow r_1 | c_2 \Rightarrow r_2 | ... | c_n \Rightarrow r_n )$.

The following table summarizes the primitive data types used by \progname{f}. 

\begin{center}
\renewcommand{\arraystretch}{1.2}
\noindent 
\begin{tabular}{l l p{7.5cm}} 
\toprule 
\textbf{Data Type} & \textbf{Notation} & \textbf{Description}\\ 
\midrule
character & char & a single symbol or digit\\
integer & $\mathbb{Z}$ & a number without a fractional component in (-$\infty$, $\infty$) \\
natural number & $\mathbb{N}$ & a number without a fractional component in [1, $\infty$) \\
real & $\mathbb{R}$ & any number in (-$\infty$, $\infty$)\\
\bottomrule
\end{tabular} 
\end{center}

\noindent
The specification of \progname{f} uses some derived data types: sequences, strings, and
tuples. Sequences are lists filled with elements of the same data type. Strings
are sequences of characters. Tuples contain a list of values, potentially of
different types. In addition, \progname{f} uses functions, which
are defined by the data types of their inputs and outputs. Local functions are
described by giving their type signature followed by their specification.

\section{Module Decomposition}

The following table is taken directly from the Module Guide document for this project.

\begin{table}[!ht]
\centering
\begin{tabular}{p{0.3\textwidth} p{0.6\textwidth}}
\toprule
\textbf{Level 1} & \textbf{Level 2}\\
\midrule

\multirow{1}{0.3\textwidth}{Hardware-Hiding Module} & \\

\midrule

\multirow{7}{0.3\textwidth}{Behaviour-Hiding Module}
  & In\\
  & Out\\
  & Params\\
  & Real Pole\\
  & Complex Pair Poles\\ 
  & Top Line Analysis\\
  & ROC\\
\midrule

{Software Decision Module} & Solver\\
\bottomrule

\end{tabular}
\caption{Module Hierarchy}
\label{tb:module}
\end{table}


\section{MIS of In} \label{sc:MIS_IN}

The secrets of this module are
 input acquisition via software \rref{rISoftware},
 validate input format \rref{rIFormat},
 validate input type \rref{rIType},
 inputs should satisfy the assumptions \rref{rAssumptions}, and
 inputs should be scaled to prevent overflow/underflow \rref{R_Inputs}.
This module handles the data structure for input coefficients and the scaling of
 the input coefficients, how the values are input (software/hardware),
 and how the values are verified (OS).
The load and verify secrets are isolated to their own methods.

\subsection{Module}

IN

\subsection{Uses}

None

\subsection{Syntax}

\begin{tabular}{p{3cm} p{2cm} p{4cm} >{\raggedright\arraybackslash}p{9cm}}
\toprule
\textbf{Name} & \textbf{In} & \textbf{Out} & \textbf{Exceptions} \\
\midrule
load\_coeffs & string & vector<double> &  FileError \\
verify\_coeffs & - & - & FileError \\
\bottomrule
\end{tabular}

\subsection{Semantics}

\subsubsection{Environment Variables}

inputFile: sequence of string \#\textit{f[i] is the ith string in the comma separated value text file f}\\ 

\subsubsection{Assumptions}

The coefficients must be ordered according to the convention from lowest order to highest order.
The number of coefficients follows. Finally the scaling follows.

\subsubsection{Access Routine Semantics}

\noindent load\_coeffs($s$):
\begin{itemize}
\item transition: The filename $s$ is first associated with the file f.  {inputFile} is used to
  modify a coefficients vector of type double, size variable of type integer, and a scaling variable
  of type double. This file can be processed by \progname{f} as each line is read.
  In this way, the option of input by software and input by hardware are handled identically.
\item verify\_coeffs() is handled by the OS plus a validation that the coefficients vector has the
  expected size.
\item exception: exc := a file name $s$ cannot be found OR the format of
  inputFile is incorrect $\Rightarrow$  FileError
\end{itemize}

\section{MIS of Out} \label{sc:MIS_OUT}

The secrets of this module are
 output via software \rref{rOSoftware},
 validate input format \rref{rOFormat},
 validate input type \rref{rOType},
This module handles the data structure for output coefficients,
 how the values are output (software/hardware),
 and how the values are verified (OS).

\subsection{Module}

OUT

\subsection{Uses}

None

\subsection{Syntax}

\begin{tabular}{p{3cm} p{2cm} p{4cm} >{\raggedright\arraybackslash}p{9cm}}
\toprule
\textbf{Name} & \textbf{In} & \textbf{Out} & \textbf{Exceptions} \\
\midrule
write\_out & double & hardware/software &  FileError \\
\bottomrule
\end{tabular}

\subsection{Semantics}

\subsubsection{Environment Variables}

outputFile: sequence of string \#\textit{f[i] is the ith string in the comma separated value text file f}\\ 

\subsubsection{Assumptions}

The radius of convergence $R_c$ and order of singularity $\mu$ must be ordered according to a convention.

\subsubsection{Access Routine Semantics}

\noindent write\_out($s$):
\begin{itemize}
\item transition: The filename $s$ is first associated with the file f.  {outputFile}. This file
  can be appended by \progname{f} as each line is read.
  In this way, the option of output by software and output by hardware are handled identically.
\item verify\_out() is handled by the OS.
\item exception: exc := a file name $s$ cannot be found $\Rightarrow$  FileError
\end{itemize}

\section{MIS of Params} \label{sc:MIS_PARAM}

The secrets of this module are parameter acquisition, format, type, distribution, and constraints.
The author is thinking of constants in a header file.

\subsection{Module}

PARAM

\subsection{Uses}

None

\subsubsection{Exported Constants}

\noindent $THREETERM\_NUSE$ of type integer. $THREETERM\_NUSE$ must be less than the size of the coefficient vector minus 10.

\noindent $SIXTERM\_NUSE$ of type integer. $SIXTERM\_NUSE$ must be less than the size of the coefficient vector minus 10.

\section{MIS of Solver} \label{sc:MIS_SOLVER}

Algorithm to find the distance to the
 nearest real pole \rref{rPoleRealSolverAlgorithm},
 nearest complex conjugate pair of poles \rref{rPoleComplexSolverAlgorithm}, and
 nearest pole in hard to resolve case \rref{rPoleComplicatedAlgorithm}.
This solver is required to solve the linear least squares problem.

\subsection{Module}

SOLVER

\subsection{Uses}

The RealPoleModule discussed in \SCref{MIS_POLEREALSOLVERMODULE},
 ComplexConjugatePoleModule discussed in \SCref{MIS_POLECOMPLEXSOLVERMODULE}, and
 the hard to resolve case module discussed in \SCref{MIS_POLETOPLINEMODULE}
 set up their respective problem for this module to resolve.

\subsection{Syntax}

The syntax depends on the pole to resolve and the algorithm used, the linear least squares
solver is the QRFactorization in all cases.

\section{MIS of Real Pole} \label{sc:MIS_POLEREALSOLVERMODULE}

Find the distance to the nearest real pole \rref{rPoleRealSolverAlgorithm}.

\subsection{Module}

REALPOLE

\subsection{Uses}

IN \SCref{MIS_IN}, OUT \SCref{MIS_OUT}, PARAM \SCref{MIS_PARAM}, SOLVER \SCref{MIS_SOLVER}

\subsection{Syntax}

\begin{tabular}{p{3cm} p{5cm} p{3cm} >{\raggedright\arraybackslash}p{5cm}}
\toprule
\textbf{Name} & \textbf{In} & \textbf{Out} & \textbf{Exceptions} \\
\midrule
  REALPOLE & vector<double>, size, scale &  $R_c$, $\mu$, error & SOLVER\_EXCEPTION \\
\bottomrule
\end{tabular}

\section{MIS of Complex Pair Poles} \label{sc:MIS_POLECOMPLEXSOLVERMODULE}

Find the distance to the nearest complex conjugate pair of poles \rref{rPoleComplexSolverAlgorithm}.

\subsection{Module}

COMPLEXPOLES

\subsection{Uses}

IN \SCref{MIS_IN}, OUT \SCref{MIS_OUT}, PARAM \SCref{MIS_PARAM}, SOLVER \SCref{MIS_SOLVER}

\subsection{Syntax}

\begin{tabular}{p{3.5cm} p{5cm} p{3cm} >{\raggedright\arraybackslash}p{5cm}}
\toprule
\textbf{Name} & \textbf{In} & \textbf{Out} & \textbf{Exceptions} \\
\midrule
  COMPLEXPOLES & vector<double>, size, scale &  $R_c$, $\mu$, error & SOLVER\_EXCEPTION \\
\bottomrule
\end{tabular}

\section{MIS of Top Line Analysis} \label{sc:MIS_POLETOPLINEMODULE}

Find the distance to the nearest complicated pole \rref{rPoleComplicatedAlgorithm}.

\subsection{Module}

TOPLINEANALYSIS

\subsection{Uses}

IN \SCref{MIS_IN}, OUT \SCref{MIS_OUT}, PARAM \SCref{MIS_PARAM}, SOLVER \SCref{MIS_SOLVER}

\subsection{Syntax}

\begin{tabular}{p{4cm} p{5cm} p{3cm} >{\raggedright\arraybackslash}p{5cm}}
\toprule
\textbf{Name} & \textbf{In} & \textbf{Out} & \textbf{Exceptions} \\
\midrule
  TOPLINEANALYSIS & vector<double>, size, scale &  $R_c$, error & SOLVER\_EXCEPTION \\
\bottomrule
\end{tabular}

\section{MIS of ROC} \label{sc:MIS_ROC}

Find the Radius of convergence $R_c$ and order of singularity $\mu$.

\subsection{Module}

ROC

\subsection{Uses}

REALPOLE \SCref{MIS_POLEREALSOLVERMODULE},
COMPLEXPOLES \SCref{MIS_POLECOMPLEXSOLVERMODULE},
TOPLINEANALYSIS \SCref{MIS_POLETOPLINEMODULE}

\subsection{Syntax}

\begin{tabular}{p{3cm} p{5cm} p{3cm} >{\raggedright\arraybackslash}p{5cm}}
\toprule
\textbf{Name} & \textbf{In} & \textbf{Out} & \textbf{Exceptions} \\
\midrule
  ROC & vector<double>, size, scale &  $R_c$, $\mu$, error & ROC\_EXCEPTION \\
\bottomrule
\end{tabular}

\bibliographystyle {plainnat}
\bibliography {../../refs/References}
\end{document}
