\documentclass[12pt, titlepage]{article}

\usepackage{booktabs}
\usepackage{tabularx}
\usepackage{hyperref}
\hypersetup{
    colorlinks,
    citecolor=blue,
    filecolor=black,
    linkcolor=red,
    urlcolor=blue
}
\usepackage[round]{natbib}

%% Comments

\usepackage{color}

\newif\ifcomments\commentsfalse

\ifcomments
\newcommand{\authornote}[3]{\textcolor{#1}{[#3 ---#2]}}
\newcommand{\todo}[1]{\textcolor{red}{[TODO: #1]}}
\else
\newcommand{\authornote}[3]{}
\newcommand{\todo}[1]{}
\fi

\newcommand{\wss}[1]{\authornote{blue}{SS}{#1}} 
\newcommand{\plt}[1]{\authornote{magenta}{TPLT}{#1}} %For explanation of the template
\newcommand{\an}[1]{\authornote{cyan}{Author}{#1}}
\newcommand{\jme}[1]{\authornote{cyan}{JME}{#1}}

% Common Parts

\usepackage{tabularx}
\usepackage{booktabs}
\usepackage{ifthen}
\usepackage{amsmath,amssymb,xspace}

%PUT YOUR PROGRAM NAME HERE %Every program should have a name
\newcommand{\progname}[1]
{%
  \ifthenelse{\equal{#1}{n}}{{\normalsize ROC\xspace}}{}%
  \ifthenelse{\equal{#1}{L}}{{\Large ROC\xspace}}{}%
  \ifthenelse{\equal{#1}{f}}{{\footnotesize ROC\xspace}}{}%
}
% Abbreviations
\newcommand{\ode}{{\footnotesize ODE}\xspace}
\newcommand{\dae}{{\footnotesize DAE}\xspace}
\newcommand{\ivp}{{\footnotesize IVP}\xspace}
\newcommand{\fadbad}{{\footnotesize FADBAD++}\xspace}
\newcommand{\plainc}{{\footnotesize C}\xspace}
\newcommand{\cpp}{{\footnotesize C++}\xspace}
\newcommand{\adolc}{{\footnotesize ADOL-C}\xspace}
\newcommand{\rdcon}{{\footnotesize DRDCV}\xspace}
\newcommand{\fortran}{{\footnotesize FORTRAN 77}\xspace}
\newcommand{\daets}{{\footnotesize DAETS}\xspace}
\newcommand{\matlab}{{\footnotesize MATLAB}\xspace}
\newcommand{\mathematica}{{\footnotesize MATHEMATICA}\xspace}
\newcommand{\maple}{{\footnotesize MAPLE}\xspace}
% User commands
\newcommand{\Quote}[1]{``{#1}''}
\newcommand{\Ni}{\noindent}
% Environment
\newcommand{\EQ}[1]{\begin{align} {#1} \end{align}}
% Mathematics
\newcommand{\parn}[1]{( {#1} )}
\newcommand{\parbg}[1]{\left(  {#1} \right)}
\newcommand{\pbg}[1]{\bigl(  {#1} \bigr)}
\newcommand{\Setbg}[1]{\bigl\{ {#1} \bigr\}}
\def\Rz{\mathbb{R}}
\def\Cz{\mathbb{C}}
\newcommand{\nliminf}[1]{\liminf\limits_{{#1} \rightarrow \infty}}
\newcommand{\nlimsup}[1]{\limsup\limits_{{#1} \rightarrow \infty}}
% Mathematics: ODE
\newcommand{\iode}{\protect{\makebox{$[t_0,\tend]$}}\xspace}
\newcommand{\lode}{\protect{\makebox{$[t_n,t_{n+1}]$}}\xspace}
\newcommand{\tend}{t_\text{end}}
\newcommand{\tc}[2]{(#1)_{#2}}
\newcommand{\Tp}{T}
\newcommand{\xn}{x_{n}}
\newcommand{\tn}{t_{n}}
% Reference
\renewcommand{\eqref}[1]{(\ref{eq:#1})}
\newcommand{\rrf}[2]{(\ref{eq:#1}--\ref{eq:#2})}
\newcommand{\chref}[1]{\ref{ch:#1}}
\newcommand{\sscref}[1]{\ref{ssc:#1}}
\newcommand{\scref}[1]{Section~\ref{sc:#1}}
\newcommand{\exref}[1]{\ref{ex:#1}}
\newcommand{\rmref}[1]{\ref{rm:#1}}
\newcommand{\apref}[1]{\ref{ap:#1}}
%\newcommand{\tbref}[1]{\ref{tb:#1}}
\newcommand{\dfref}[1]{\ref{df:#1}}
\newcommand{\leref}[1]{\ref{le:#1}}
\newcommand{\fgref}[1]{\ref{fg:#1}}
\newcommand{\coref}[1]{\ref{co:#1}}
\newcommand{\thref}[1]{\ref{th:#1}}
\newcommand{\agref}[1]{\ref{ag:#1}}
\newcommand{\asref}[1]{\ref{as:#1}}
\newcommand{\EQref}[1]{Equation~(\ref{eq:#1})}
\newcommand{\CHref}[1]{Chapter~\ref{ch:#1}}
\newcommand{\SSCref}[1]{Subsection~\ref{ssc:#1}}
\newcommand{\SCref}[1]{Section~\ref{sc:#1}}
\newcommand{\EXref}[1]{Example~\ref{ex:#1}}
\newcommand{\RMref}[1]{Remark~\ref{rm:#1}}
\newcommand{\APref}[1]{Appendix~\ref{ap:#1}}
\newcommand{\TBref}[1]{Table~\ref{tb:#1}}
\newcommand{\DFref}[1]{Definition~\ref{df:#1}}
\newcommand{\LEref}[1]{Lemma~\ref{le:#1}}
\newcommand{\FGref}[1]{Figure~\ref{fg:#1}}
\newcommand{\COref}[1]{Corollary~\ref{co:#1}}
\newcommand{\THref}[1]{Theorem~\ref{th:#1}}
\newcommand{\AGref}[1]{Algorithm~\ref{ag:#1}}
\newcommand{\ASref}[1]{Assumption~\ref{as:#1}}



\begin{document}

\title{Project Title: Unit Verification and Validation Plan for \progname{}} 
\author{Author Name}
\date{\today}
	
\maketitle

\pagenumbering{roman}

\section{Revision History}

\begin{tabularx}{\textwidth}{p{3cm}p{2cm}X}
\toprule {\bf Date} & {\bf Version} & {\bf Notes}\\
\midrule
Date 1 & 1.0 & Notes\\
Date 2 & 1.1 & Notes\\
\bottomrule
\end{tabularx}

~\newpage

\tableofcontents

\listoftables

\wss{Do not include if not relevant}

\listoffigures

\wss{Do not include if not relevant}

\newpage

\section{Symbols, Abbreviations and Acronyms}

\renewcommand{\arraystretch}{1.2}
\begin{tabular}{l l} 
  \toprule		
  \textbf{symbol} & \textbf{description}\\
  \midrule 
  T & Test\\
  \bottomrule
\end{tabular}\\

\wss{symbols, abbreviations or acronyms -- you can reference the SRS
  \citep{SRS}, MG or MIS tables if needed}

\newpage

\pagenumbering{arabic}

This document ... \wss{provide an introductory blurb and roadmap of the
  unit V\&V plan}

\wss{If content in this document is already covered in the VnV System plan, you
  do not have to repeat it.}

\section{General Information}

\subsection{Purpose}

\wss{Identify software that is being unit tested (verified).}

\subsection{Scope}

\wss{What modules are outside of the scope.  If there are modules that are
  developed by someone else, then you would say here if you aren't planning on
  verifying them.  There may also be modules that are part of your software, but
  have a lower priority for verification than others.  If this is the case,
  explain your rationale for the ranking of module importance.}

\section{Plan}
	
\subsection{Verification and Validation Team}

\wss{You, your classmates and the course instructor.  Maybe your supervisor.}

\subsection{Automated Testing and Verification Tools}

\wss{What tools are you using for automated testing.  Likely a unit testing
  framework and maybe a profiling tool, like ValGrind.  Other possible tools
  include a static analyzer, make, continuous integration tools, test coverage
  tools, etc.  Explain your plans for summarizing code coverage metrics.
  Linters are another important class of tools.  For the programming language
  you select, you should look at the available linters.  There may also be tools
  that verify that coding standards have been respected, like flake9 for
  Python.}

\subsection{Non-Testing Based Verification}

\wss{List any approaches like code inspection, code walkthrough, symbolic
  execution etc.  Enter not applicable if you do not plan on any non-testing
  based verification.}

\section{Unit Test Description}

\wss{Reference your MIS and explain your overall philosophy for test case
  selection.}

\subsection{Tests for Functional Requirements}

\wss{Most of the verification will be through automated unit testing.  If
  appropriate specific modules can be verified by a non-testing based
  technique.  That can also be documented in this section.}

\subsubsection{Module 1}

\wss{Include a blurb here to explain why the subsections below cover the module.
  References to the MIS would be good.  You will want tests from a black box
  perspective and from a white box perspective.  Explain to the reader how the
  tests were selected.}

\begin{enumerate}

\item{test-id1\\}

Type: \wss{Functional, Dynamic, Manual, Automatic, Static etc. Most will
  be automatic}
					
Initial State: 
					
Input: 
					
Output: \wss{The expected result for the given inputs}

Test Case Derivation: \wss{Justify the expected value given in the Output field}

How test will be performed: 
					
\item{test-id2\\}

Type: \wss{Functional, Dynamic, Manual, Automatic, Static etc. Most will
  be automatic}
					
Initial State: 
					
Input: 
					
Output: \wss{The expected result for the given inputs}

Test Case Derivation: \wss{Justify the expected value given in the Output field}

How test will be performed: 

\item{...\\}
    
\end{enumerate}

\subsubsection{Module 2}

...

\subsection{Tests for Nonfunctional Requirements}

\wss{If there is a module that needs to be independently assessed for
  performance, those test cases can go here.  In some projects, planning for
  nonfunctional tests of units will not be that relevant.}

\wss{These tests may involve collecting performance data from previously
  mentioned functional tests.}

\subsubsection{Module ?}
		
\begin{enumerate}

\item{test-id1\\}

Type: \wss{Functional, Dynamic, Manual, Automatic, Static etc. Most will
  be automatic}
					
Initial State: 
					
Input/Condition: 
					
Output/Result: 
					
How test will be performed: 
					
\item{test-id2\\}

Type: Functional, Dynamic, Manual, Static etc.
					
Initial State: 
					
Input: 
					
Output: 
					
How test will be performed: 

\end{enumerate}

\subsubsection{Module ?}

...

\subsection{Traceability Between Test Cases and Modules}

\wss{Provide evidence that all of the modules have been considered.}

\bibliographystyle{plainnat}

\bibliography{../../../refs/References}

\newpage

\section{Appendix}

\wss{This is where you can place additional information, as appropriate}

\subsection{Symbolic Parameters}

\wss{The definition of the test cases may call for SYMBOLIC\_CONSTANTS.
Their values are defined in this section for easy maintenance.}

\end{document}