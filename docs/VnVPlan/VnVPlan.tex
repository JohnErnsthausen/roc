\documentclass[12pt, titlepage]{article}

\usepackage{amsmath, mathtools}
\usepackage{amsfonts}
\usepackage{amssymb}
\usepackage{booktabs}
\usepackage{tabularx}
\usepackage{xspace}

\usepackage{graphicx}
\usepackage{colortbl}
\usepackage{xr}
\usepackage{longtable}
\usepackage{xfrac}
\usepackage{float}
\usepackage{siunitx}
\usepackage{caption}
\usepackage{pdflscape}
\usepackage{afterpage}

\usepackage{fullpage}
\usepackage[round]{natbib}
\usepackage{multirow}
%\usepackage{refcheck}
\usepackage{lipsum}

\usepackage[section]{placeins}
\usepackage{array}

%% Comments

\usepackage{color}

\newif\ifcomments\commentsfalse

\ifcomments
\newcommand{\authornote}[3]{\textcolor{#1}{[#3 ---#2]}}
\newcommand{\todo}[1]{\textcolor{red}{[TODO: #1]}}
\else
\newcommand{\authornote}[3]{}
\newcommand{\todo}[1]{}
\fi

\newcommand{\wss}[1]{\authornote{blue}{SS}{#1}} 
\newcommand{\plt}[1]{\authornote{magenta}{TPLT}{#1}} %For explanation of the template
\newcommand{\an}[1]{\authornote{cyan}{Author}{#1}}
\newcommand{\jme}[1]{\authornote{cyan}{JME}{#1}}

% Common Parts

\usepackage{tabularx}
\usepackage{booktabs}
\usepackage{ifthen}
\usepackage{amsmath,amssymb,xspace}

%PUT YOUR PROGRAM NAME HERE %Every program should have a name
\newcommand{\progname}[1]
{%
  \ifthenelse{\equal{#1}{n}}{{\normalsize ROC\xspace}}{}%
  \ifthenelse{\equal{#1}{L}}{{\Large ROC\xspace}}{}%
  \ifthenelse{\equal{#1}{f}}{{\footnotesize ROC\xspace}}{}%
}
% Abbreviations
\newcommand{\ode}{{\footnotesize ODE}\xspace}
\newcommand{\dae}{{\footnotesize DAE}\xspace}
\newcommand{\ivp}{{\footnotesize IVP}\xspace}
\newcommand{\fadbad}{{\footnotesize FADBAD++}\xspace}
\newcommand{\plainc}{{\footnotesize C}\xspace}
\newcommand{\cpp}{{\footnotesize C++}\xspace}
\newcommand{\adolc}{{\footnotesize ADOL-C}\xspace}
\newcommand{\rdcon}{{\footnotesize DRDCV}\xspace}
\newcommand{\fortran}{{\footnotesize FORTRAN 77}\xspace}
\newcommand{\daets}{{\footnotesize DAETS}\xspace}
\newcommand{\matlab}{{\footnotesize MATLAB}\xspace}
\newcommand{\mathematica}{{\footnotesize MATHEMATICA}\xspace}
\newcommand{\maple}{{\footnotesize MAPLE}\xspace}
% User commands
\newcommand{\Quote}[1]{``{#1}''}
\newcommand{\Ni}{\noindent}
% Environment
\newcommand{\EQ}[1]{\begin{align} {#1} \end{align}}
% Mathematics
\newcommand{\parn}[1]{( {#1} )}
\newcommand{\parbg}[1]{\left(  {#1} \right)}
\newcommand{\pbg}[1]{\bigl(  {#1} \bigr)}
\newcommand{\Setbg}[1]{\bigl\{ {#1} \bigr\}}
\def\Rz{\mathbb{R}}
\def\Cz{\mathbb{C}}
\newcommand{\nliminf}[1]{\liminf\limits_{{#1} \rightarrow \infty}}
\newcommand{\nlimsup}[1]{\limsup\limits_{{#1} \rightarrow \infty}}
% Mathematics: ODE
\newcommand{\iode}{\protect{\makebox{$[t_0,\tend]$}}\xspace}
\newcommand{\lode}{\protect{\makebox{$[t_n,t_{n+1}]$}}\xspace}
\newcommand{\tend}{t_\text{end}}
\newcommand{\tc}[2]{(#1)_{#2}}
\newcommand{\Tp}{T}
\newcommand{\xn}{x_{n}}
\newcommand{\tn}{t_{n}}
% Reference
\renewcommand{\eqref}[1]{(\ref{eq:#1})}
\newcommand{\rrf}[2]{(\ref{eq:#1}--\ref{eq:#2})}
\newcommand{\chref}[1]{\ref{ch:#1}}
\newcommand{\sscref}[1]{\ref{ssc:#1}}
\newcommand{\scref}[1]{Section~\ref{sc:#1}}
\newcommand{\exref}[1]{\ref{ex:#1}}
\newcommand{\rmref}[1]{\ref{rm:#1}}
\newcommand{\apref}[1]{\ref{ap:#1}}
%\newcommand{\tbref}[1]{\ref{tb:#1}}
\newcommand{\dfref}[1]{\ref{df:#1}}
\newcommand{\leref}[1]{\ref{le:#1}}
\newcommand{\fgref}[1]{\ref{fg:#1}}
\newcommand{\coref}[1]{\ref{co:#1}}
\newcommand{\thref}[1]{\ref{th:#1}}
\newcommand{\agref}[1]{\ref{ag:#1}}
\newcommand{\asref}[1]{\ref{as:#1}}
\newcommand{\EQref}[1]{Equation~(\ref{eq:#1})}
\newcommand{\CHref}[1]{Chapter~\ref{ch:#1}}
\newcommand{\SSCref}[1]{Subsection~\ref{ssc:#1}}
\newcommand{\SCref}[1]{Section~\ref{sc:#1}}
\newcommand{\EXref}[1]{Example~\ref{ex:#1}}
\newcommand{\RMref}[1]{Remark~\ref{rm:#1}}
\newcommand{\APref}[1]{Appendix~\ref{ap:#1}}
\newcommand{\TBref}[1]{Table~\ref{tb:#1}}
\newcommand{\DFref}[1]{Definition~\ref{df:#1}}
\newcommand{\LEref}[1]{Lemma~\ref{le:#1}}
\newcommand{\FGref}[1]{Figure~\ref{fg:#1}}
\newcommand{\COref}[1]{Corollary~\ref{co:#1}}
\newcommand{\THref}[1]{Theorem~\ref{th:#1}}
\newcommand{\AGref}[1]{Algorithm~\ref{ag:#1}}
\newcommand{\ASref}[1]{Assumption~\ref{as:#1}}



\makeatletter
\newcommand*{\addFileDependency}[1]{% argument=file name and extension
  \typeout{(#1)}
  \@addtofilelist{#1}
  \IfFileExists{#1}{}{\typeout{No file #1.}}
}
\makeatother

\newcommand*{\myexternaldocument}[1]{%
    \externaldocument{#1}%
    \addFileDependency{#1.tex}%
    \addFileDependency{#1.aux}%
}

\externaldocument{../SRS/SRS}

\newcommand{\rref}[1]{(R\ref{#1})}

\begin{document}

\title{System Verification and Validation Plan for \progname{L}:
Software estimating the radius of convergence of a power series} 
\author{John M Ernsthausen}
\date{\today}
	
\maketitle

~\newpage

\pagenumbering{roman}

\tableofcontents

%\listoftables

%\listoffigures

~\newpage

\section*{Revision History}

\begin{tabularx}{\textwidth}{p{4cm}p{2cm}X}
\toprule {\bf Date} & {\bf Version} & {\bf Notes}\\
\midrule
29 October 2020 & 1.0 & First submission\\
24 December 2020 & 2.0 & Second submission\\
\bottomrule
\end{tabularx}

~\newpage

\section{Symbols, Abbreviations, and Acronyms}

Symbols, abbreviations, and acronyms applicable to \progname{f} are enumerated
in Section 1 of the Software Requirements Document (SRS) \citep{SRS}.

\newpage

\pagenumbering{arabic}

% \input examples
% \input lagrangianfacility
% \input mechfacility
% \input vec3d
% \input fcn 
% \printindex

%CHAPTER
\section{Introduction} \label{sc:introduction}

%  Possibly move to SRS or introduction of MIS
%
%
%  Construct the series centered at $z_0 \in \Rz$
%  \EQ
%  {
%    \label{eq:power-series}
%    \sum_{n=0}^{\infty} c_n (z-z_0)^n
%  }
%  from a sequence $\Setbg{c_n}$ of real numbers where the $n^\text{th}$ term in the sequence corresponds
%  to the $n^\text{th}$ coefficient in the series. We associate a sequence $\Setbg{s_n}$ of partial sums
%  \EQ
%  {
%    \label{eq:partial-sum}
%    s_n \defeq \sum_{k=0}^n c_k (z-z_0)^k
%  }
%  with the power series. If $\Setbg{s_n} \rightarrow s$ as $n \rightarrow \infty$,
%  then we say $\Setbg{s_n}$ converges to $s$. The number $s$ is the sum of the series, and
%  we write $s$ as \eqref{power-series}. If $\Setbg{s_n}$ diverges, then the series is said to diverge.
%  
%  We cannot perform an infinite sum on a digital computer. However, given a tolerance \tol
%  and a convergent power series, there exists an integer $N$ such that, for all
%  $m \geq n \geq N$, $| \sum_{k=n}^{m} c_k | < \tol$. We assume that we know $N$. Our software
%  \progname{f} will estimate the radius of convergence $R_c$ from the first $N$ terms 
%  in the power series. The coefficients may be scaled with a scaling $h$ to
%  prevent numerical overflow. The default scaling is $h=1$. Scaling the coefficients
%  is a change of variables $v(z) \defeq (z - z_0)/h$ in \eqref{power-series}. With the scaled coefficients
%   $\tilde c_n \defeq c_n h^n$ and the change of variables, \eqref{power-series} transforms to
%  \EQ
%  {
%    \label{eq:power-series-scaled}
%    \sum_{n=0}^{\infty} \tilde c_n v^n.
%  }
%  When $r_c$ is the radius of the circle of convergence of \eqref{power-series-scaled}, $R_c = h r_c$
%  is the radius of the circle of convergence of \eqref{power-series}.
%  \progname{f} may not compute $R_c$ exactly. These are the assumptions under which the \progname{f} software
%  operates. In the sequel, we denote both scaled and unscaled coefficients by $c_n$.
%  
%  \cite{chang1982} observed that the coefficients of \eqref{power-series} follow a few
%  very definite patterns characterized by the location of primary singularities. Real valued power series
%  can only have poles, logarithmic branch points, and essential singularities. Moreover, these singularities
%  occur on the real axis or in complex conjugate pairs. The effects of secondary singularities disappear
%  whenever sufficiently long power series are used. To determine $R_c$ and the order of the singularity $\mu$,
%  \cite{chang1982} fit a given finite sequence to a model.
%  
%  Recall that a primary singularity of \eqref{power-series} is the closest singularity to the series
%  expansion point $z_0$ in the complex plane. All other singularities are secondary singularities.

This document provides a verification and validation plan for developing \progname{f}.

%CHAPTER
\section{General Information}

The scope of this \progname{f} project is limited to top-line analysis.
Top-line analysis always applies to any power series \eqref{power-series}. It resolves situations
where secondary singularities are less distinguishable from primary singularities. However it is
less accurate. It does have a convergence analysis \citep{chang1982}.

Also add 3TA and 6TA.

% Move to MIS
%
%
%  \subsection{Summary}
%  
%  In \cite{SRS}, we detailed the algorithm implementing top-line analysis.
%  
%  \paragraph{Module findrc} \label{findrc}
%  We know the tolerance $\tol$, the scale $h$, the integer $N$, and the sequence
%  $\Setbg{c_0, c_1, \ldots, c_{N}}$ by assumption, as in \SCref{introduction}.
%  We extract the last $15$ terms from the sequence.
%  With $k = i+N-14$, obtain the best linear fit $y(k) = m k + b$ in the $2$-norm to the points
%  \EQ
%  {
%    \label{eq:approximation-points}
%    \Setbg{(N-14, \log_{10} | c_{N-14} |), (N-13, \log_{10} |c_{N-13}|), \ldots, (N, \log_{10} |c_{N}|)},
%  }
%  that is, find $m$ and $b$ such that $\sum_{i=0}^{14} |\log_{10} |c_{i+N-14}| - y(i+N-14)|^2$ is minimized.
%  The approximation problem \eqref{approximation-points} is the well known
%  {\it linear least-squares problem} \cite{GoVL89}.
%  The model parameter $m$ will be negative whenever the series \eqref{power-series} converges and
%  positive whenever it diverges.The radius of convergence $R_c = h/10^m$.
%  
%  \paragraph{Module findmu}
%  Start with the series resulting from integration of the given series three times and
%  fit the coefficients with Module {\bf findrc}. If that graph is linear, meaning the
%  minimizer has norm less than \tol, then the slope is accepted and the order of the
%  singularity is 3. If the graph opens upward, then the series is differentiated term-wise
%  to reduce the second derivative of the graph, and a new top-line fit is computed with
%  Module {\bf findrc}. This process is repeated, reducing $\mu$ by 1 each time, until the
%  graph opens downward or until seven term-wise differentiations have been tested.
%  If seven term-wise differentiations have been tested and each result in turn proves
%  unsatisfactory, then the final estimate for $R_c$ is reduced by 10 percent for a
%  conservative estimate for $R_c$ and $\mu=-4$ is returned.
%  
%  We implement Module {\bf findrc} and Module {\bf findrc} in \progname{f}.
%  This document provides plan for verification and validation \progname{f} while implementing
%  these modules in \cpp.

\subsection{Objectives}

Our objective it to implement Module {\bf findrc} and Module {\bf findrc} via
{\it test driving} in \cpp. The implementation must satisfy the requirements
enumerated in \cite{SRS}.

Assume that the assumptions are satisfied. Recall the requirements.

%  \noindent \begin{itemize}
%  
%  \item[R\refstepcounter{reqnum}\thereqnum \label{R_Inputs}:] The input
%    sequence should be scaled by $h$ to prevent overflow/underflow.
%    If this is not possible, \progname{f} will find a scale.
%  
%  \item[R\refstepcounter{reqnum}\thereqnum \label{R_Calculate}:]
%    \progname{f} should execute as fast as the \cite{chang1982} software \rdcon.
%   A NONFUNCTIONAL requirement!
%  
%  \item[R\refstepcounter{reqnum}\thereqnum \label{R_VerifyOutput}:]
%    \progname{f} should estimate the radius of convergence $R_c$ and
%      the order of singularity $\mu$ for the following cases where $R_c$ and $\mu$ are known.
%  
%    The real valued function $1/(z-z_0)^{\mu}$.
%  
%    The real valued function $1/(1 + 25*(z-z_0)^2)^{\mu}$.
%  
%    Top line analysis while solving \dae \ivp by the TS method should verify
%      the computed step is within the circle of convergence and compare
%      with the \cite{chang1982} algorithm \rdcon.
%  
%  \item[R\refstepcounter{reqnum}\thereqnum \label{R_Output}:] We must not overestimate $R_c$.
%    If $R_c$ is overestimated, then the power-series is a divergence sum on the overestimation.
%  
%    In \ode solving by TS methods, underestimating $R_c$ is acceptable as an underestimation
%    results in a slight increase in computational effort for solving an \ode \ivp.
%  \end{itemize}

In a nutshell these requirements validate and verify \progname{f}.

\subsection{Relevant Documentation}

Relevant documentation includes the authors Software Requirements (SRS) Document \citep{SRS}, the
authors Module Guide (MG, to be written), and the authors Module Interface Specification (MIS, to be written).

\section{Verification and Validation Plan}

Verification and Validation of \progname{f} includes automated testing at the module level,
the system level, and integration level. This document will additionally propose continuous integration.

\subsection{Verification and Validation Team}

The \progname{f} team includes
author
\href{https://github.com/JohnErnsthausen}{John Ernsthausen},
fellow students
\href{https://github.com/LeilaMousapour}{Leila Mousapour},
\href{https://github.com/salahhessien}{Salah Gamal aly Hessien},
\href{https://github.com/liziscool}{Liz Hofer},
and
\href{https://github.com/XingzhiMac}{Xingzhi Liu} as well as Professors
\href{http://baraksh.com}{Barak Shoshany},
\href{https://github.com/smiths}{Spencer Smith},
\href{https://www.cs.mu.edu/~george/}{George Corliss},
and
\href{http://www.cas.mcmaster.ca/~nedialk}{Ned Nedialkov}.
The author appreciates the helpful comments and superior guidance on this project.

\subsection{SRS Verification Plan}

The SRS document Verification and Validation Plan for \progname{f} will be peer-reviewed by
domain expert Leila Mousapour and secondary reviewer Salah Gamal aly Hessien.
Prof. Spencer Smith the course instructor and my supervisor Ned Nedialkov will review the SRS
document.

The SRS document will be published to \href{https://github.com/JohnErnsthausen/roc}{GitHub}.
Defects will be addressed with issues on the GitHub platform.

\subsection{Design Verification Plan}

The Design documents MG and MIS plan for \progname{f} will be peer-reviewed by
domain expert Leila Mousapour and secondary reviewer Xingzhi Liu.
Prof. Spencer Smith the course instructor and my supervisor Ned Nedialkov will also review these Design
documents.

The MG and MIS documents will be published to \href{https://github.com/JohnErnsthausen/roc}{GitHub}.
Defects will be addressed with issues on the GitHub platform.

\subsection{Implementation Verification Plan}

Dynamic testing through unit tests (Gtest/GMock), memory testing (valgrind), profiling (gprof),
and code coverage (gcov).

Static testing though linting (clang-format).

Continuous integration through GitHub, TravisCI, and Coveralls.

All verification will be automatic.

\subsubsection{Dynamic testing}
Dynamic testing requires the program to be executed. For example, unit test cases
are run, and the results are checked against expected behaviour.

\paragraph{Unit testing (Gtest/GMock)}:
The Implementation Verification Plan for \progname{f} includes automated testing at the module level.
For testing at the module level, the author plans to follow the Test Driven Development
software development practices as described by \cite{langr2013} for {\it test driving}.
Test driving is an interactive process for software development where tests are determined
as part of the software development process. However on a high level, some unit tests
will be offered in the sequel.

Let us recall test driving \cite{langr2013}.
Test driving results in unit tests. A unit test verifies the {\it behavior} of a code unit,
where a {\it unit} is the smallest testable piece of an application.

A single unit test consists of a descriptive name and a series of code statements conceptually
divided into four parts

\begin{enumerate}
  \item (Optional) statements that set up a context for execution
  \item One or more statements to invoke the behavior to be verified
  \item One or more statements to verify the expected outcome
  \item (Optional) cleanup statements
\end{enumerate}

The first three parts are referred to as {\it Given-When-Then}. {\it Given} a context, {\it When}
the test executes some code, {\it Then} some behavior is verified.

Test Driven Development is used to test drive new behavior into the code in small increments. To
add a new piece of behavior into the system, first write a test to define that behavior. The
existence of a test that will not pass drives the developer to implement the corresponding behavior.
The increment should be the smallest meaningful amount of code, one or two lines of code and one assertion.

\paragraph{Memory testing (valgrind):}
Code free of memory leaks.

\paragraph{Profiling (gprof):}
Number of method calls and time spent in each method. Used to help performance.

\paragraph{Code coverage (gcov):}
Reveals the code used/not used. Percentage of code used per directory and per file.

\subsubsection{Static testing}
Static testing does not involve program execution.
Testing techniques simulate the dynamic environment.
Static testing includes syntax checking.

\paragraph{Linting (clang-format):}
The code should be well formatted. Formats can be customized in XXX.

\subsubsection{Continuous integration}
\begin{itemize}
\item Information available on
  \href{https://en.wikipedia.org/wiki/Continuous_integration}{Wikipedia}
\item Developers integrate their code into a shared repo frequently (multiple
  times a day)
\item Each integration is automatically accompanied by regression tests and
  other build tasks
\item Build server
\begin{itemize}
\item Unit tests
\item Integration tests
\item Static analysis
\item Profile performance
\item Extract documentation
\item Update project web-page
\item Portability tests
\item etc.
\end{itemize}
\item Avoids potentially extreme problems with integration when the baseline and
  a developer's code greatly differ
\end{itemize}

\begin{itemize}
\item Eliminates the ``it works on my machine'' problem
\item Package dependencies with your apps
\item A container for lightweight virtualization
\item Not a full VM
\end{itemize}

Continuous integration through GitHub, TravisCI, and Coveralls.

\subsubsection{Parallel testing}
In parallel testing, one compares to other programs.

\progname{f} includes automated testing at the system level
and integration level. The code output
will be compared with the output of \rdcon developed by \cite{chang1982} for validation at
the integration level.

\subsubsection{Comparison with closed-form solution}
Yes I do this too, in my unit tests.


\subsection{Automated Testing and Verification Tools}

Automated Testing and Verification Tools will be extensively used in the development of \progname{f}
for automation at the module level, the system level, and integration level.
These tools include
\href{https://git.kernel.org/pub/scm/git/git.git}{git} a distributed version-control system
for tracking changes in source code during software development,
\href{https://cmake.org/}{cmake} for build automation,
\href{https://github.com/google/googletest}{gtest}
as a unit testing framework, \href{http://clang.llvm.org/docs/ClangFormat.html}{clang-format}
for consistent code style formatting, and \href{https://www.valgrind.org}{valgrind}
to identify memory leaks.

This document will additionally propose to branch out into three new areas for the author.
In the DevOps arena,
continuous integration will be pursued with \href{https://travis-ci.org}{TravisCI}.
The continuous integration tool should send an email confirming the success state
of integrating the new code into production.
The author will explore the application of linters and metrics for code coverage.

\subsection{Software Validation Plan}

Validation is the validation of the requirements. Validation compares experimental
data to output from \progname{f} to confirm or reject the problem model. Validation
considers the applicability of the equations and assumptions to the problem space.
However, \progname{f} is not modelling a physical problem. Thus a
Software Validation Plan is not applicable to this project.

\section{System Test Description}

System tests are about the public interface.

Sample Functional System Testing
\begin{itemize}
\item Requirements: Determines if the system can perform its function correctly and that the correctness can be sustained over a continuous period of time
\item Regression: Determines if changes to the system do not invalidate previous positive testing results
\item Error Handling: Determines the ability of the system to properly process
  incorrect transactions
\item Manual Support: Determines that the manual support procedures are documented and complete, where manual support involves procedures, interfaces between people and the system, and training procedures
\item Parallel: Determines the results of the new application are consistent with the processing of the previous application or version of the application
\end{itemize}

Sample Nonfunctional System Testing
\begin{itemize}
\item Stress testing - Determines if the system can function when subject to large volumes
\item Usability testing
\item Performance measurement
\end{itemize}

\subsection{Tests for Functional Requirements}

\subsubsection{Parallel testing}

\progname{f} includes automated testing at the system level
and integration level. The code output
will be compared with the output of \rdcon developed by \cite{chang1982} for validation
at the integration level.

And see Nonfunctional tests.

\subsection{Tests for Nonfunctional Requirements}

\subsubsection{Timing}

Requirement \rref{R_Calculate} says that \progname{f} should execute as fast as
the \cite{chang1982} software \rdcon. The following tests represent a comparison between
\progname{f} and \rdcon.

\paragraph{Proportion of time spent finding the stepsize}

\begin{enumerate}

  \item{Layne-Watson \citep{watson1979}\\}

% Control: 
					
% Initial State: 
					
Input: Load a file with the Taylor series solution of the Layne-Watson problem. Each time the
    local initial value problem was solved there will be a Taylor series of length $N$ and a scale $h$,
    which are the required inputs for \progname{f} and \rdcon.
					
Output: Difference between time for \progname{f} to solve the problem and time for \rdcon to
    solve the problem.

Test Case Derivation: Two techniques resolving the same data.
					
How test will be performed: Automatically.
					
\item{Planetary-Motion \citep{enright1987examples}\\}

% Control: 
					
% Initial State: 
					
Input: Load a file with the Taylor series solution of the Planetary-Motion problem. Each time the
    local initial value problem was solved there will be a Taylor series of length $N$ and a scale $h$,
    which are the required inputs for \progname{f} and \rdcon.
					
Output: Difference between time for \progname{f} to solve problem and time for \rdcon to solve problem.

Test Case Derivation: Two techniques resolving the same data.
					
How test will be performed: Automatically.

\end{enumerate}

\subsubsection{Accuracy}

Requirement \rref{R_VerifyOutput} says that the new top line
analysis in \progname{f} should compute a $R_c$ and $\mu$ which is
close to the values computed with the \cite{chang1982} algorithm \rdcon.
This test treats \rdcon as a pseudo oracle.
The comparison will be carried out on the process of solving a \dae \ivp by the TS method.

\paragraph{Accuracy in finding the stepsize}

\begin{enumerate}

\item{Layne-Watson \citep{watson1979}\\}

% Control: 
					
% Initial State: 
					
Input: Load a file with the Taylor series solution of the Layne-Watson problem. Each time the
    local initial value problem was solved there will be a Taylor series of length $N$ and a scale $h$,
    which are the required inputs for \progname{f} and \rdcon.
					
Output: Each time the local initial value problem was solved, find $R_c$ and $\mu$ with
    \progname{f} and \rdcon. Compare the results. Expect the results to compare within 10\%.

Test Case Derivation: Two techniques resolving the same data.
					
How test will be performed: Automatically.
					
\item{Planetary-Motion \citep{enright1987examples}\\}

% Control: 
					
% Initial State: 
					
Input: Load a file with the Taylor series solution of the Planetary-Motion problem. Each time the
    local initial value problem was solved there will be a Taylor series of length $N$ and a scale $h$,
    which are the required inputs for \progname{f} and \rdcon.
					
Output: Each time the local initial value problem was solved, find $R_c$ and $\mu$ with
    \progname{f} and \rdcon. Compare the results. Expect the results to compare within 10\%.

Test Case Derivation: Two techniques resolving the same data.
					
How test will be performed: Automatically.

\end{enumerate}

% \subsection{Tests for Nonfunctional Requirements}
% 
% \wss{The nonfunctional requirements for accuracy will likely just reference the
%   appropriate functional tests from above.  The test cases should mention
%   reporting the relative error for these tests.}
% 
% \wss{Tests related to usability could include conducting a usability test and
%   survey.}
% 
% \subsubsection{Area of Testing1}
% 		
% \paragraph{Title for Test}
% 
% \begin{enumerate}
% 
% \item{test-id1\\}
% 
% Type: 
% 					
% Initial State: 
% 					
% Input/Condition: 
% 					
% Output/Result: 
% 					
% How test will be performed: 
% 					
% \item{test-id2\\}
% 
% Type: Functional, Dynamic, Manual, Static etc.
% 					
% Initial State: 
% 					
% Input: 
% 					
% Output: 
% 					
% How test will be performed: 
% 
% \end{enumerate}
% 
% \subsubsection{Area of Testing2}
% 
% ...

\subsection{Traceability Between Test Cases and Requirements}

In each test, the requirement supported by the test case is stated.

However I would find a traceability table difficult to present as I have 158 tests, difficult to maintain, and a duplication of work. I wouldn't find this table useful. Here's why.

In the Module Guide there is "Table 2: Trace Between Requirements and Modules". Here I'm thinking of Module as a Parnas Module, a work order. For every work order, I Test Drive the behavior given in the work order into my code base. All of the behaviors required by the module must have corresponding testing code. I then do code coverage on my tests. If I have 100\% code coverage of my tests and all the required behaviors are covered in the module. Then by my testing, I automatically have the functionality of the traceability table in discussion making the traceability table unnecessary work.

The Software Development theories of Test Driving recommend running all test and observing that all tests pass. The interest of the third party in "Let’s say a third party person is only interested in testing the R3" should be highly discouraged.

\section{Unit Test Description}

\subsubsection{Input testing}
% Issue #9

User misuses the inputs: Length should be negative. Ensure the program gives an error.

Input testing in these input requirements:
\rref{rIHardware} Input acquisition via hardware.
\rref{rISoftware} Input acquisition via software.
\rref{rIFormat} Validate input format.
\rref{rIType} Validate input type.

%  \wss{Reference your MIS and explain your overall philosophy for test case
%    selection.}  
%  \wss{This section should not be filled in until after the MIS has
%    been completed.}
%  
%  
%  
%  
%  %  \item[R\refstepcounter{reqnum}\thereqnum \label{R_Inputs}:] The input
%  %    sequence should be scaled by $h$ to prevent overflow/underflow.
%  %    If this is not possible, \progname{f} will find a scale.
%  %  
%  %  \item[R\refstepcounter{reqnum}\thereqnum \label{R_Calculate}:]
%  %    \progname{f} should execute as fast as the \cite{chang1982} software \rdcon.
%  %  
%  %  \item[R\refstepcounter{reqnum}\thereqnum \label{R_VerifyOutput}:]
%  %    \progname{f} should estimate the radius of convergence $R_c$ and
%  %      the order of singularity $\mu$ for the following cases where $R_c$ and $\mu$ are known.
%  %  
%  %    The real valued function $1/(z-z_0)^{\mu}$.
%  %  
%  %    The real valued function $1/(1 + 25*(z-z_0)^2)^{\mu}$.
%  %  
%  %    Top line analysis while solving \dae \ivp by the TS method should verify
%  %      the computed step is within the circle of convergence and compare
%  %      with the \cite{chang1982} algorithm \rdcon.
%  %  
%  %  \item[R\refstepcounter{reqnum}\thereqnum \label{R_Output}:] We must not overestimate $R_c$.
%  %    If $R_c$ is overestimated, then the power-series is a divergence sum on the overestimation.
%  %  
%  %    In \ode solving by TS methods, underestimating $R_c$ is acceptable as an underestimation
%  %    results in a slight increase in computational effort for solving an \ode \ivp.
%  %  \end{itemize}
%  \subsection{Unit Testing Scope}
%  
%  \wss{What modules are outside of the scope.  If there are modules that are
%    developed by someone else, then you would say here if you aren't planning on
%    verifying them.  There may also be modules that are part of your software, but
%    have a lower priority for verification than others.  If this is the case,
%    explain your rationale for the ranking of module importance.}
%  
%  \subsection{Tests for Functional Requirements}
%  
%  \wss{Most of the verification will be through automated unit testing.  If
%    appropriate specific modules can be verified by a non-testing based
%    technique.  That can also be documented in this section.}
%  
%  \subsubsection{Module 1}
%  
%  \wss{Include a blurb here to explain why the subsections below cover the module.
%    References to the MIS would be good.  You will want tests from a black box
%    perspective and from a white box perspective.  Explain to the reader how the
%    tests were selected.}
%  
%  \begin{enumerate}
%  
%  \item{test-id1\\}
%  
%  Type: \wss{Functional, Dynamic, Manual, Automatic, Static etc. Most will
%    be automatic}
%  					
%  Initial State: 
%  					
%  Input: 
%  					
%  Output: \wss{The expected result for the given inputs}
%  
%  Test Case Derivation: \wss{Justify the expected value given in the Output field}
%  
%  How test will be performed: 
%  					
%  \item{test-id2\\}
%  
%  Type: \wss{Functional, Dynamic, Manual, Automatic, Static etc. Most will
%    be automatic}
%  					
%  Initial State: 
%  					
%  Input: 
%  					
%  Output: \wss{The expected result for the given inputs}
%  
%  Test Case Derivation: \wss{Justify the expected value given in the Output field}
%  
%  How test will be performed: 
%  
%  \item{...\\}
%      
%  \end{enumerate}
%  
%  \subsubsection{Module 2}
%  
%  ...
%  
%  \subsection{Tests for Nonfunctional Requirements}
%  
%  \wss{If there is a module that needs to be independently assessed for
%    performance, those test cases can go here.  In some projects, planning for
%    nonfunctional tests of units will not be that relevant.}
%  
%  \wss{These tests may involve collecting performance data from previously
%    mentioned functional tests.}
%  
%  \subsubsection{Module ?}
%  		
%  \begin{enumerate}
%  
%  \item{test-id1\\}
%  
%  Type: \wss{Functional, Dynamic, Manual, Automatic, Static etc. Most will
%    be automatic}
%  					
%  Initial State: 
%  					
%  Input/Condition: 
%  					
%  Output/Result: 
%  					
%  How test will be performed: 
%  					
%  \item{test-id2\\}
%  
%  Type: Functional, Dynamic, Manual, Static etc.
%  					
%  Initial State: 
%  					
%  Input: 
%  					
%  Output: 
%  					
%  How test will be performed: 
%  
%  \end{enumerate}
%  
%  \subsubsection{Module ?}
%  
%  ...
%  
%  \subsection{Traceability Between Test Cases and Modules}
%  
%  \wss{Provide evidence that all of the modules have been considered.}
				
\bibliographystyle{plainnat}

\bibliography{../../refs/References}

%  \newpage
%  
%  \section{Appendix}
%  
%  This is where you can place additional information.
%  
%  \subsection{Symbolic Parameters}
%  
%  The definition of the test cases will call for SYMBOLIC\_CONSTANTS.
%  Their values are defined in this section for easy maintenance.
%  
%  \subsection{Usability Survey Questions?}
%  
%  \wss{This is a section that would be appropriate for some projects.}

\end{document}
