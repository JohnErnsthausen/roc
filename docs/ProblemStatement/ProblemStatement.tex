\documentclass{article}

% Common Parts

\usepackage{tabularx}
\usepackage{booktabs}
\usepackage{ifthen}
\usepackage{amsmath,amssymb,xspace}

%PUT YOUR PROGRAM NAME HERE %Every program should have a name
\newcommand{\progname}[1]
{%
  \ifthenelse{\equal{#1}{n}}{{\normalsize ROC\xspace}}{}%
  \ifthenelse{\equal{#1}{L}}{{\Large ROC\xspace}}{}%
  \ifthenelse{\equal{#1}{f}}{{\footnotesize ROC\xspace}}{}%
}
% Abbreviations
\newcommand{\ode}{{\footnotesize ODE}\xspace}
\newcommand{\dae}{{\footnotesize DAE}\xspace}
\newcommand{\ivp}{{\footnotesize IVP}\xspace}
\newcommand{\fadbad}{{\footnotesize FADBAD++}\xspace}
\newcommand{\plainc}{{\footnotesize C}\xspace}
\newcommand{\cpp}{{\footnotesize C++}\xspace}
\newcommand{\adolc}{{\footnotesize ADOL-C}\xspace}
\newcommand{\rdcon}{{\footnotesize DRDCV}\xspace}
\newcommand{\fortran}{{\footnotesize FORTRAN 77}\xspace}
\newcommand{\daets}{{\footnotesize DAETS}\xspace}
\newcommand{\matlab}{{\footnotesize MATLAB}\xspace}
\newcommand{\mathematica}{{\footnotesize MATHEMATICA}\xspace}
\newcommand{\maple}{{\footnotesize MAPLE}\xspace}
% User commands
\newcommand{\Quote}[1]{``{#1}''}
\newcommand{\Ni}{\noindent}
% Environment
\newcommand{\EQ}[1]{\begin{align} {#1} \end{align}}
% Mathematics
\newcommand{\parn}[1]{( {#1} )}
\newcommand{\parbg}[1]{\left(  {#1} \right)}
\newcommand{\pbg}[1]{\bigl(  {#1} \bigr)}
\newcommand{\Setbg}[1]{\bigl\{ {#1} \bigr\}}
\def\Rz{\mathbb{R}}
\def\Cz{\mathbb{C}}
\newcommand{\nliminf}[1]{\liminf\limits_{{#1} \rightarrow \infty}}
\newcommand{\nlimsup}[1]{\limsup\limits_{{#1} \rightarrow \infty}}
% Mathematics: ODE
\newcommand{\iode}{\protect{\makebox{$[t_0,\tend]$}}\xspace}
\newcommand{\lode}{\protect{\makebox{$[t_n,t_{n+1}]$}}\xspace}
\newcommand{\tend}{t_\text{end}}
\newcommand{\tc}[2]{(#1)_{#2}}
\newcommand{\Tp}{T}
\newcommand{\xn}{x_{n}}
\newcommand{\tn}{t_{n}}
% Reference
\renewcommand{\eqref}[1]{(\ref{eq:#1})}
\newcommand{\rrf}[2]{(\ref{eq:#1}--\ref{eq:#2})}
\newcommand{\chref}[1]{\ref{ch:#1}}
\newcommand{\sscref}[1]{\ref{ssc:#1}}
\newcommand{\scref}[1]{Section~\ref{sc:#1}}
\newcommand{\exref}[1]{\ref{ex:#1}}
\newcommand{\rmref}[1]{\ref{rm:#1}}
\newcommand{\apref}[1]{\ref{ap:#1}}
%\newcommand{\tbref}[1]{\ref{tb:#1}}
\newcommand{\dfref}[1]{\ref{df:#1}}
\newcommand{\leref}[1]{\ref{le:#1}}
\newcommand{\fgref}[1]{\ref{fg:#1}}
\newcommand{\coref}[1]{\ref{co:#1}}
\newcommand{\thref}[1]{\ref{th:#1}}
\newcommand{\agref}[1]{\ref{ag:#1}}
\newcommand{\asref}[1]{\ref{as:#1}}
\newcommand{\EQref}[1]{Equation~(\ref{eq:#1})}
\newcommand{\CHref}[1]{Chapter~\ref{ch:#1}}
\newcommand{\SSCref}[1]{Subsection~\ref{ssc:#1}}
\newcommand{\SCref}[1]{Section~\ref{sc:#1}}
\newcommand{\EXref}[1]{Example~\ref{ex:#1}}
\newcommand{\RMref}[1]{Remark~\ref{rm:#1}}
\newcommand{\APref}[1]{Appendix~\ref{ap:#1}}
\newcommand{\TBref}[1]{Table~\ref{tb:#1}}
\newcommand{\DFref}[1]{Definition~\ref{df:#1}}
\newcommand{\LEref}[1]{Lemma~\ref{le:#1}}
\newcommand{\FGref}[1]{Figure~\ref{fg:#1}}
\newcommand{\COref}[1]{Corollary~\ref{co:#1}}
\newcommand{\THref}[1]{Theorem~\ref{th:#1}}
\newcommand{\AGref}[1]{Algorithm~\ref{ag:#1}}
\newcommand{\ASref}[1]{Assumption~\ref{as:#1}}



\title{CAS 741: Problem Statement\\Radius of Convergence}

\author{John Ernsthausen (macid: ernsthjm)}

\date{}

%% Comments

\usepackage{color}

\newif\ifcomments\commentsfalse

\ifcomments
\newcommand{\authornote}[3]{\textcolor{#1}{[#3 ---#2]}}
\newcommand{\todo}[1]{\textcolor{red}{[TODO: #1]}}
\else
\newcommand{\authornote}[3]{}
\newcommand{\todo}[1]{}
\fi

\newcommand{\wss}[1]{\authornote{blue}{SS}{#1}} 
\newcommand{\plt}[1]{\authornote{magenta}{TPLT}{#1}} %For explanation of the template
\newcommand{\an}[1]{\authornote{cyan}{Author}{#1}}
\newcommand{\jme}[1]{\authornote{cyan}{JME}{#1}}


\begin{document}

\maketitle

\begin{table}[hp]
\caption{Revision History} \label{TblRevisionHistory}
\begin{tabularx}{\textwidth}{llX}
\toprule
\textbf{Date} & \textbf{Developer(s)} & \textbf{Change}\\
\midrule
September 18, 2020 & John Ernsthausen & Initial draft\\
September 25, 2020 & John Ernsthausen & Removed connection to \daets\\
\bottomrule
\end{tabularx}
\end{table}

After generating a real-valued Taylor series (TS) approximate solution of order $p$ to the
ordinary differential equation (\ode) initial-value problem (\ivp) 
\EQ
{
  \label{eq:introduction-mathematical-odeivp}
  y'\parn{t} = f\pbg{t,y\parn{t}},
  \quad
  y\parn{t_0} = y_0 \in \mathcal D \subset \Rz^d,
  \quad
  t \in \iode \subset \Rz,
}
the TS methods defined in
Jorba and Zou \cite{jorba2005software} (JZ),
Bergsma and Mooij \cite{bergsma2016application} (BM),
and
Chang and Corliss \cite{chang1982} (CC)
explicitly require an estimate for the TS radius of convergence (RC).

At each $\parn{t_n,x_n}$, $n \geq 0$, the TS method for the numerical solution
of \eqref{introduction-mathematical-odeivp} computes Taylor coefficients (TCs)
$\tc{\xn}{i}$ at $t_{n}$ to construct the TS approximate solution
\begin{align}
  \label{eq:tss}
  \Tp(t) = \xn + \sum_{i=1}^{p}\tc{\xn}{i}(t-\tn)^{i} \quad\mbox{on}\quad \lode.
\end{align}
TCs can be readily computed through automatic differentiation (AD) with packages
such as \fadbad \cite{FADBAD++} and \adolc \cite{GriewankADBook2/e}.
In choosing the order $p$ and the stepsize $h = t_{n+1} - t_n$, the goal is to minimize the amount
of computational work required during the integration process while maintaining a user specified accuracy tolerance. 

JZ compute the RC from the TS as the minimum of the $p-1$ and $p$ terms in the usual ratio test.
While the JZ process is straightforward, it can be inaccurate.
CC fit the tail of the computed TS to the TS of a model problem, which has proven to be satisfactory
on standard test problems \cite{enright1987examples}.

I propose to estimate the RC of a given Taylor series based on top line analysis
\cite[pp.~127--128]{chang1982}, one of several sub-algorithms of the CC algorithm.
Top line analysis is the default approach when all their other tests fail.

%  \wss{comment}
%  
%  You can also leave comments for yourself, like this:
%  
%  \an{comment}

\bibliographystyle{elsarticle-num} 
\bibliography{Bib}
\end{document}
