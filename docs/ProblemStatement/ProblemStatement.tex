\documentclass[12pt]{article}

\usepackage{amsmath, mathtools}
\usepackage{amsfonts}
\usepackage{amssymb}
\usepackage{booktabs}
\usepackage{tabularx}
\usepackage{xspace}

\usepackage{graphicx}
\usepackage{colortbl}
\usepackage{xr}
\usepackage{longtable}
\usepackage{xfrac}
\usepackage{float}
\usepackage{siunitx}
\usepackage{caption}
\usepackage{geometry}
\usepackage{pdflscape}
\usepackage{afterpage}

\usepackage{fullpage}
\usepackage[round]{natbib}
%\usepackage{refcheck}
\usepackage{lipsum}

%% Comments

\usepackage{color}

\newif\ifcomments\commentsfalse

\ifcomments
\newcommand{\authornote}[3]{\textcolor{#1}{[#3 ---#2]}}
\newcommand{\todo}[1]{\textcolor{red}{[TODO: #1]}}
\else
\newcommand{\authornote}[3]{}
\newcommand{\todo}[1]{}
\fi

\newcommand{\wss}[1]{\authornote{blue}{SS}{#1}} 
\newcommand{\plt}[1]{\authornote{magenta}{TPLT}{#1}} %For explanation of the template
\newcommand{\an}[1]{\authornote{cyan}{Author}{#1}}
\newcommand{\jme}[1]{\authornote{cyan}{JME}{#1}}

% Common Parts

\usepackage{tabularx}
\usepackage{booktabs}
\usepackage{ifthen}
\usepackage{amsmath,amssymb,xspace}

%PUT YOUR PROGRAM NAME HERE %Every program should have a name
\newcommand{\progname}[1]
{%
  \ifthenelse{\equal{#1}{n}}{{\normalsize ROC\xspace}}{}%
  \ifthenelse{\equal{#1}{L}}{{\Large ROC\xspace}}{}%
  \ifthenelse{\equal{#1}{f}}{{\footnotesize ROC\xspace}}{}%
}
% Abbreviations
\newcommand{\ode}{{\footnotesize ODE}\xspace}
\newcommand{\dae}{{\footnotesize DAE}\xspace}
\newcommand{\ivp}{{\footnotesize IVP}\xspace}
\newcommand{\fadbad}{{\footnotesize FADBAD++}\xspace}
\newcommand{\plainc}{{\footnotesize C}\xspace}
\newcommand{\cpp}{{\footnotesize C++}\xspace}
\newcommand{\adolc}{{\footnotesize ADOL-C}\xspace}
\newcommand{\rdcon}{{\footnotesize DRDCV}\xspace}
\newcommand{\fortran}{{\footnotesize FORTRAN 77}\xspace}
\newcommand{\daets}{{\footnotesize DAETS}\xspace}
\newcommand{\matlab}{{\footnotesize MATLAB}\xspace}
\newcommand{\mathematica}{{\footnotesize MATHEMATICA}\xspace}
\newcommand{\maple}{{\footnotesize MAPLE}\xspace}
% User commands
\newcommand{\Quote}[1]{``{#1}''}
\newcommand{\Ni}{\noindent}
% Environment
\newcommand{\EQ}[1]{\begin{align} {#1} \end{align}}
% Mathematics
\newcommand{\parn}[1]{( {#1} )}
\newcommand{\parbg}[1]{\left(  {#1} \right)}
\newcommand{\pbg}[1]{\bigl(  {#1} \bigr)}
\newcommand{\Setbg}[1]{\bigl\{ {#1} \bigr\}}
\def\Rz{\mathbb{R}}
\def\Cz{\mathbb{C}}
\newcommand{\nliminf}[1]{\liminf\limits_{{#1} \rightarrow \infty}}
\newcommand{\nlimsup}[1]{\limsup\limits_{{#1} \rightarrow \infty}}
% Mathematics: ODE
\newcommand{\iode}{\protect{\makebox{$[t_0,\tend]$}}\xspace}
\newcommand{\lode}{\protect{\makebox{$[t_n,t_{n+1}]$}}\xspace}
\newcommand{\tend}{t_\text{end}}
\newcommand{\tc}[2]{(#1)_{#2}}
\newcommand{\Tp}{T}
\newcommand{\xn}{x_{n}}
\newcommand{\tn}{t_{n}}
% Reference
\renewcommand{\eqref}[1]{(\ref{eq:#1})}
\newcommand{\rrf}[2]{(\ref{eq:#1}--\ref{eq:#2})}
\newcommand{\chref}[1]{\ref{ch:#1}}
\newcommand{\sscref}[1]{\ref{ssc:#1}}
\newcommand{\scref}[1]{Section~\ref{sc:#1}}
\newcommand{\exref}[1]{\ref{ex:#1}}
\newcommand{\rmref}[1]{\ref{rm:#1}}
\newcommand{\apref}[1]{\ref{ap:#1}}
%\newcommand{\tbref}[1]{\ref{tb:#1}}
\newcommand{\dfref}[1]{\ref{df:#1}}
\newcommand{\leref}[1]{\ref{le:#1}}
\newcommand{\fgref}[1]{\ref{fg:#1}}
\newcommand{\coref}[1]{\ref{co:#1}}
\newcommand{\thref}[1]{\ref{th:#1}}
\newcommand{\agref}[1]{\ref{ag:#1}}
\newcommand{\asref}[1]{\ref{as:#1}}
\newcommand{\EQref}[1]{Equation~(\ref{eq:#1})}
\newcommand{\CHref}[1]{Chapter~\ref{ch:#1}}
\newcommand{\SSCref}[1]{Subsection~\ref{ssc:#1}}
\newcommand{\SCref}[1]{Section~\ref{sc:#1}}
\newcommand{\EXref}[1]{Example~\ref{ex:#1}}
\newcommand{\RMref}[1]{Remark~\ref{rm:#1}}
\newcommand{\APref}[1]{Appendix~\ref{ap:#1}}
\newcommand{\TBref}[1]{Table~\ref{tb:#1}}
\newcommand{\DFref}[1]{Definition~\ref{df:#1}}
\newcommand{\LEref}[1]{Lemma~\ref{le:#1}}
\newcommand{\FGref}[1]{Figure~\ref{fg:#1}}
\newcommand{\COref}[1]{Corollary~\ref{co:#1}}
\newcommand{\THref}[1]{Theorem~\ref{th:#1}}
\newcommand{\AGref}[1]{Algorithm~\ref{ag:#1}}
\newcommand{\ASref}[1]{Assumption~\ref{as:#1}}



% For easy change of table widths
\newcommand{\colZwidth}{1.0\textwidth}
\newcommand{\colAwidth}{0.13\textwidth}
\newcommand{\colBwidth}{0.82\textwidth}
\newcommand{\colCwidth}{0.1\textwidth}
\newcommand{\colDwidth}{0.05\textwidth}
\newcommand{\colEwidth}{0.8\textwidth}
\newcommand{\colFwidth}{0.17\textwidth}
\newcommand{\colGwidth}{0.5\textwidth}
\newcommand{\colHwidth}{0.28\textwidth}

% Used so that cross-references have a meaningful prefix
\newcounter{defnum} %Definition Number
\newcommand{\dthedefnum}{GD\thedefnum}
\newcommand{\dref}[1]{GD\ref{#1}}
\newcounter{datadefnum} %Datadefinition Number
\newcommand{\ddthedatadefnum}{DD\thedatadefnum}
\newcommand{\ddref}[1]{DD\ref{#1}}
\newcounter{theorynum} %Theory Number
\newcommand{\tthetheorynum}{T\thetheorynum}
\newcommand{\tref}[1]{TM\ref{#1}}
\newcounter{tablenum} %Table Number
\newcommand{\tbthetablenum}{T\thetablenum}
\newcommand{\tbref}[1]{TB\ref{#1}}
\newcounter{assumpnum} %Assumption Number
\newcommand{\atheassumpnum}{P\theassumpnum}
\newcommand{\aref}[1]{A\ref{#1}}
\newcounter{goalnum} %Goal Number
\newcommand{\gthegoalnum}{P\thegoalnum}
\newcommand{\gsref}[1]{GS\ref{#1}}
\newcounter{instnum} %Instance Number
\newcommand{\itheinstnum}{IM\theinstnum}
\newcommand{\iref}[1]{IM\ref{#1}}

\newcounter{reqnum} %Requirement Number
\newcommand{\rthereqnum}{R\thereqnum}
\newcommand{\rref}[1]{R\ref{#1}}
\newcommand{\rlabel}[1]{\refstepcounter{reqnum} \rthereqnum \label{#1}:}

\newcounter{nfrnum} %NFR Number
\newcommand{\rthenfrnum}{NFR\thenfrnum}
\newcommand{\nfrref}[1]{NFR\ref{#1}}

\newcounter{lcnum} %Likely change number
\newcommand{\lthelcnum}{LC\thelcnum}
\newcommand{\lcref}[1]{LC\ref{#1}}

\begin{document}

\title{CAS 741: Problem Statement\\Radius of Convergence}
\author{John Ernsthausen (macid: ernsthjm)}
\date{\today}
	
\maketitle

\begin{table}[hp]
\caption{Revision History} \label{TblRevisionHistory}
\begin{tabularx}{\textwidth}{llX}
\toprule
\textbf{Date} & \textbf{Developer(s)} & \textbf{Change}\\
\midrule
September 18, 2020 & John Ernsthausen & Initial draft\\
September 25, 2020 & John Ernsthausen & Removed connection to \daets\\
24 December 2020 & John Ernsthausen & Second submission\\
\bottomrule
\end{tabularx}
\end{table}

Construct the power series centered at $z_0 \in \Rz$
\EQ
{
  \label{eq:power-series}
  \sum_{n=0}^{\infty} c_n (z-z_0)^n
}
from a sequence $\Setbg{c_n}$ of real numbers where the $n^\text{th}$ term in the sequence corresponds
to the $n^\text{th}$ coefficient in the series. We associate a sequence $\Setbg{s_n}$ of partial sums
\EQ
{
  \label{eq:partial-sum}
  s_n \defeq \sum_{k=0}^n c_k (z-z_0)^k
}
with the power series. If $\Setbg{s_n} \rightarrow s$ as $n \rightarrow \infty$,
then we say $\Setbg{s_n}$ converges to $s$. The number $s$ is the sum of the series, and
we write $s$ as \eqref{power-series}. If $\Setbg{s_n}$ diverges, then the series is said to diverge.

We cannot perform an infinite sum on a digital computer, and we cannot identify the set of all $z$
near $z_0$ such that \eqref{power-series} converges by exhaustive checking. We need to identify
the circle of convergence by its radius.
A power series always converges inside a circle of convergence of radius $R_c$.

Chang and Corliss \cite{chang1982} observed that the coefficients of \eqref{power-series} follow a few
very definite patterns characterized by the location of primary singularities. Real valued power series
can only have poles, logarithmic branch points, and essential singularities. Moreover, these singularities
occur on the real axis or in complex conjugate pairs. The effects of secondary singularities disappear
whenever sufficiently long power series are used.
Recall that a primary singularity of \eqref{power-series} is the closest singularity to the series
expansion point $z_0$ in the complex plane. All other singularities are secondary singularities.

To determine $R_c$ and the order of the singularity $\mu$,
Chang and Corliss \cite{chang1982} fit a given finite sequence corresponding to the partial sum $s_N$
for $N$ sufficiently large to a model.

I propose to estimate the $R_c$ of a given power series based on the three term analysis
for a real pole, six term analysis for a pair of complex conjugate poles, and top line analysis
\cite[pp.~127--128]{chang1982}. The input is a finite sequence of scaled coefficients
and their scaling. The output is $R_c$ and the order of the singularity $\mu$.

%  We cannot perform an infinite sum on a digital computer. However, given a tolerance \tol
%  and a convergent power series, there exists an integer $N$ such that, for all
%  $m \geq n \geq N$, $| \sum_{k=n}^{m} c_k | < \tol$. We assume that we know $N$. Our software
%  \progname{f} will estimate the radius of convergence $R_c$ from the first $N$ terms 
%  in the power series. The coefficients may be scaled with a scaling $h$ to
%  prevent numerical overflow. The default scaling is $h=1$. Scaling the coefficients
%  is a change of variables $v(z) \defeq (z - z_0)/h$ in \eqref{power-series}. With the scaled coefficients
%   $\tilde c_n \defeq c_n h^n$ and the change of variables, \eqref{power-series} transforms to
%  \EQ
%  {
%    \label{eq:power-series-scaled}
%    \sum_{n=0}^{\infty} \tilde c_n v^n.
%  }
%  When $r_c$ is the radius of the circle of convergence of \eqref{power-series-scaled}, $R_c = h r_c$
%  is the radius of the circle of convergence of \eqref{power-series}.
%  \progname{f} may not compute $R_c$ exactly. These are the assumptions under which the \progname{f} software
%  operates. In the sequel, we denote both scaled and unscaled coefficients by $c_n$.
%  
%  \cite{chang1982} observed that the coefficients of \eqref{power-series} follow a few
%  very definite patterns characterized by the location of primary singularities. Real valued power series
%  can only have poles, logarithmic branch points, and essential singularities. Moreover, these singularities
%  occur on the real axis or in complex conjugate pairs. The effects of secondary singularities disappear
%  whenever sufficiently long power series are used. To determine $R_c$ and the order of the singularity $\mu$,
%  \cite{chang1982} fit a given finite sequence to a model.
%  
%  Recall that a primary singularity of \eqref{power-series} is the closest singularity to the series
%  expansion point $z_0$ in the complex plane. All other singularities are secondary singularities.
%  
%  I propose to estimate the RC of a given Taylor series based on top line analysis
%  \cite[pp.~127--128]{chang1982}, one of several sub-algorithms of the CC algorithm.
%  Top line analysis is the default approach when all their other tests fail.

\bibliographystyle {plainnat}
\bibliography {../../refs/References}

\end{document}
