\documentclass{article}

%% Common Parts

\newcommand{\progname}{roc} % PUT YOUR PROGRAM NAME HERE %Every program
                                 % should have a name


\title{CAS 741: Problem Statement\\Radius of Convergence}

\author{John Ernsthausen (macid: ernsthjm)}

\date{}

\input{../Comments}

\begin{document}

\maketitle

\begin{table}[hp]
\caption{Revision History} \label{TblRevisionHistory}
\begin{tabularx}{\textwidth}{llX}
\toprule
\textbf{Date} & \textbf{Developer(s)} & \textbf{Change}\\
\midrule
September 18, 2020 & John Ernsthausen & Initial draft\\
September 25, 2020 & John Ernsthausen & Removed connection to \daets\\
\bottomrule
\end{tabularx}
\end{table}

After generating a real-valued Taylor series (TS) approximate solution of order $p$ to the
ordinary differential equation (\ode) initial-value problem (\ivp) 
\EQ
{
  \label{eq:introduction-mathematical-odeivp}
  y'\parn{t} = f\pbg{t,y\parn{t}},
  \quad
  y\parn{t_0} = y_0 \in \mathcal D \subset \Rz^d,
  \quad
  t \in \iode \subset \Rz,
}
the TS methods defined in
Jorba and Zou \cite{jorba2005software} (JZ),
Bergsma and Mooij \cite{bergsma2016application} (BM),
and
Chang and Corliss \cite{chang1982} (CC)
explicitly require an estimate for the TS radius of convergence (RC).

At each $\parn{t_n,x_n}$, $n \geq 0$, the TS method for the numerical solution
of \eqref{introduction-mathematical-odeivp} computes Taylor coefficients (TCs)
$\tc{\xn}{i}$ at $t_{n}$ to construct the TS approximate solution
\begin{align}
  \label{eq:tss}
  \Tp(t) = \xn + \sum_{i=1}^{p}\tc{\xn}{i}(t-\tn)^{i} \quad\mbox{on}\quad \lode.
\end{align}
TCs can be readily computed through automatic differentiation (AD) with packages
such as \fadbad \cite{FADBAD++} and \adolc \cite{GriewankADBook2/e}.
In choosing the order $p$ and the stepsize $h = t_{n+1} - t_n$, the goal is to minimize the amount
of computational work required during the integration process while maintaining a user specified accuracy tolerance. 

JZ compute the RC from the TS as the minimum of the $p-1$ and $p$ terms in the usual ratio test.
While the JZ process is straightforward, it can be inaccurate.
CC fit the tail of the computed TS to the TS of a model problem, which has proven to be satisfactory
on standard test problems \cite{enright1987examples}.

I propose to estimate the RC of a given Taylor series based on top line analysis
\cite[pp.~127--128]{chang1982}, one of several sub-algorithms of the CC algorithm.
Top line analysis is the default approach when all their other tests fail.

%  \wss{comment}
%  
%  You can also leave comments for yourself, like this:
%  
%  \an{comment}

\bibliographystyle{elsarticle-num} 
\bibliography{Bib}
\end{document}
