\documentclass[12pt]{article}

\usepackage{enumitem}

\usepackage{amssymb}
\usepackage{amsfonts}
\usepackage{amsmath}

\usepackage{hyperref}
\hypersetup{colorlinks=true,
    linkcolor=blue,
    citecolor=blue,
    filecolor=blue,
    urlcolor=blue,
    unicode=false}
\urlstyle{same}

\newlist{todolist}{itemize}{2}
\setlist[todolist]{label=$\square$}
\usepackage{pifont}
\newcommand{\cmark}{\ding{51}}%
\newcommand{\xmark}{\ding{55}}%
\newcommand{\done}{\rlap{$\square$}{\raisebox{2pt}{\large\hspace{1pt}\cmark}}%
\hspace{-2.5pt}}
\newcommand{\wontfix}{\rlap{$\square$}{\large\hspace{1pt}\xmark}}

\begin{document}

\title{Writing Checklist}
\author{Spencer Smith}
\date{\today}

\maketitle

% Show an item is done by   \item[\done] Frame the problem
% Show an item will not be fixed by   \item[\wontfix] profit

\begin{itemize}
  
\item \LaTeX{} points
  \begin{todolist}
  \item Only tex file (and possibly pdf files, or image files) are under version
    control (\texttt{aux} files etc. are not under version control)
    \item Opening and closing ``quotes'' are used (\verb|``quotes''|)
  \item Periods that do not end sentences are followed by only one space:
    \verb|``I like Dr.\ Smith.''|, or for no linebreaks: \verb|``I like Dr.~Smith.''|
  \item Long names in math mode use either mathit or text, or equivalent: $coeff$
    (\verb|$coeff$|) versus $\mathit{coeff}$ (\verb|$\mathit{coeff}$|) versus
    $\text{coeff}$ (\verb|$\text{coeff}$|).
  \item For CAS 741 the instructor would like to be able to add comments to your
    tex files.  Please be sure to include the \texttt{Comments.tex} file in your repo and
    in your tex files.
  \item Text lines should be 80 characters wide.  That is, the text has a hard-wrap at
    80 characters.  This is done to facilitate meaingful diffs between different
    commits.  (Some ideas on how to do this are given below.)
  \item Template comments (plt) do not show in the pdf version, either by
    removing them, or by turning them off.
  \item References and labels are used so that maintenance is feasible
  \item Cross-references between documents are used as appropriate
  \item BibTeX is used for generating bibliographic references
  \end{todolist}

\item Structure
  \begin{todolist}
  \item There is always some text between section headings
  \item There aren't instances of only one subsection within a section
  \end{todolist}

\item Spelling, Grammar and attention to detail
  \begin{todolist}
  \item Each punctuation symbol (period, comma, semicolon, question mark,
    exclamation point) has no space before it.
  \item Opening parentheses (brackets) have a space before, closing parentheses
    have a space after the symbol.
  \item Parentheses (brackets) occur in pairs, one opening and one closing
  \item All sentences begin with a capital letter.
  \item Document is spell checked!
  \item Grammar has been checked (review, ask someone else to review (at least a few
    sections)).
  \item That and which are used correctly
    (\url{http://www.kentlaw.edu/academics/lrw/grinker/LwtaThat_Versus_Which.htm})
  \item Symbols used outside of a formula should be formatted the same way as
    they are in the equation.  For instance, when listing the variables in an
    equation, you should still use math mode for the symbols.
  \item Include a \texttt{.gitignore} file in your repo so that generated files
    are ignored by git.  More information is available on-line on
    \href{https://en.wikipedia.org/wiki/Hidden_file_and_hidden_directory}
    {Hidden files and hidden directories}.
  \item All hyperlinks work
  \item Every figure has a caption
  \item Every table has a heading
  \item All acronyms are expanded on their first usage, using capitals to
    show the source of each letter in the acronym.  Defining the acronym only in
    a table at the beginning of the document is not enough.
  \end{todolist}

\item Avoid Low Information Content phrases 
    (\href{https://www.webpages.uidaho.edu/range357/extra-refs/empty-words.htm}{List
      of LIC phrases})

  \begin{todolist}
  \item ``in order to'' simplified to ``to''
  \item ...
  \end{todolist}

\item Writing style
  \begin{todolist}
  \item Avoid sentences that start with ``It.''
  \item Paragraphs are structured well (clear topic sentence, cohesive)
  \item Paragraphs are concise (not wordy)

  \end{todolist}
  
\end{itemize}

\subsubsection*{Fixed Width \LaTeX{} Text}

Having the \LaTeX{} text at a fixed width (hard-wrap) is useful when the source
is under version control.  The fixed line lengths help with isolating the
changes between diffs.

Although the checklist mentions an 80 column width, any reasonable fixed width
is fine.

The hard-wrap shouldn't be done manually.  Most editors will have some facility
for fixed width.  In emacs it is called auto-fill.  Some advice from previous
and current students:

\begin{itemize}
\item In TEXMaker, you can do: User $>$ Run script $>$ hardwordwrap
\item Wrapping is easy in VSCode, Emacs, and Vim
\end{itemize}
\end{document}
