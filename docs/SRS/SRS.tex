\documentclass[12pt]{article}

\usepackage{amsmath, mathtools}
\usepackage{amsfonts}
\usepackage{amssymb}
\usepackage{booktabs}
\usepackage{tabularx}
\usepackage{xspace}

\usepackage{graphicx}
\usepackage{colortbl}
\usepackage{xr}
\usepackage{longtable}
\usepackage{xfrac}
\usepackage{float}
\usepackage{siunitx}
\usepackage{caption}
\usepackage{geometry}
\usepackage{pdflscape}
\usepackage{afterpage}

\usepackage{fullpage}
\usepackage[round]{natbib}
%\usepackage{refcheck}
\usepackage{lipsum}

\input{../Comments}
%% Common Parts

\newcommand{\progname}{roc} % PUT YOUR PROGRAM NAME HERE %Every program
                                 % should have a name


% For easy change of table widths
\newcommand{\colZwidth}{1.0\textwidth}
\newcommand{\colAwidth}{0.13\textwidth}
\newcommand{\colBwidth}{0.82\textwidth}
\newcommand{\colCwidth}{0.1\textwidth}
\newcommand{\colDwidth}{0.05\textwidth}
\newcommand{\colEwidth}{0.8\textwidth}
\newcommand{\colFwidth}{0.17\textwidth}
\newcommand{\colGwidth}{0.5\textwidth}
\newcommand{\colHwidth}{0.28\textwidth}

% Used so that cross-references have a meaningful prefix
\newcounter{defnum} %Definition Number
\newcommand{\dthedefnum}{GD\thedefnum}
\newcommand{\dref}[1]{GD\ref{#1}}
\newcounter{datadefnum} %Datadefinition Number
\newcommand{\ddthedatadefnum}{DD\thedatadefnum}
\newcommand{\ddref}[1]{DD\ref{#1}}
\newcounter{theorynum} %Theory Number
\newcommand{\tthetheorynum}{T\thetheorynum}
\newcommand{\tref}[1]{TM\ref{#1}}
\newcounter{tablenum} %Table Number
\newcommand{\tbthetablenum}{T\thetablenum}
\newcommand{\tbref}[1]{TB\ref{#1}}
\newcounter{assumpnum} %Assumption Number
\newcommand{\atheassumpnum}{P\theassumpnum}
\newcommand{\aref}[1]{A\ref{#1}}
\newcounter{goalnum} %Goal Number
\newcommand{\gthegoalnum}{P\thegoalnum}
\newcommand{\gsref}[1]{GS\ref{#1}}
\newcounter{instnum} %Instance Number
\newcommand{\itheinstnum}{IM\theinstnum}
\newcommand{\iref}[1]{IM\ref{#1}}

\newcounter{reqnum} %Requirement Number
\newcommand{\rthereqnum}{R\thereqnum}
\newcommand{\rref}[1]{R\ref{#1}}
\newcommand{\rlabel}[1]{\refstepcounter{reqnum} \rthereqnum \label{#1}:}

\newcounter{nfrnum} %NFR Number
\newcommand{\rthenfrnum}{NFR\thenfrnum}
\newcommand{\nfrref}[1]{NFR\ref{#1}}

\newcounter{lcnum} %Likely change number
\newcommand{\lthelcnum}{LC\thelcnum}
\newcommand{\lcref}[1]{LC\ref{#1}}

\begin{document}

\title{Software Requirements Specification for \progname{L}: Software estimating the radius of convergence of
a power series} 
\author{John M Ernsthausen}
\date{\today}
	
\maketitle

~\newpage

\pagenumbering{roman}

\tableofcontents

~\newpage

\section*{Revision History}

\begin{tabularx}{\textwidth}{p{4cm}p{2cm}X}
\toprule {\bf Date} & {\bf Version} & {\bf Notes}\\
\midrule
12 October 2020 & 1.0 & First submission\\
24 December 2020 & 2.0 & Second submission\\
\bottomrule
\end{tabularx}

~\newpage

\section{Reference Material}

This section records information for easy reference.

\subsection{Table of Units}

A Table of Units is not applicable to \progname{f}.

\subsection{Table of Symbols}

The table that follows summarizes the mathematical notation used in this document.

\renewcommand{\arraystretch}{1.2}
%\noindent \begin{tabularx}{1.0\textwidth}{l l X}
\noindent \begin{longtable*}{l p{12cm}} \toprule
\textbf{symbol} & \textbf{description}\\
\midrule
  $z \in \Cz$ & A member $z$ of the complex numbers $\Cz$\\
  $x \in \Rz$ & A member $x$ of the real numbers $\Rz$\\
  $\Setbg{c_n} \subset \Cz$ & A sequence of complex numbers
  whose $n^\text{th}$ term is $c_n \in \Cz$\\
  $\Setbg{c_n}_{n=0}^{N-1} \subset \Cz$ & A finite sequence of $N$ complex numbers
  whose $n^\text{th}$ term is $c_n \in \Cz$\\
  $\sum_{n=0}^{\infty} c_n (z-z_0)^n$ & A power series. $c_n \in \Cz$ is the
  $n^\text{th}$ coefficient. $z^n$ is the $n^\text{th}$ power of $z \in \Cz$. $z_0$ is the center point.\\ 
  $\nliminf{n}$ & Lower subsequential limit\\
  $\nlimsup{n}$ & Upper subsequential limit\\
  $R_c$ & Radius of the circle of convergence\\
  $\Rz^d$ & A $d$-dimension real vector space\\
  $\mathcal D$ & An open subset of $\Rz^d$\\
  $[a,b] \subset \Rz$ & Interval of real numbers, $t \in [a,b]$ means $a \leq t \leq b$ and $t \in \Rz$\\
  $\tc{\xn}{n}$ & $n^\text{th}$ Taylor coefficient of $t \in \Rz \mapsto x \in \Rz^d$ evaluated at $t_n \in \Rz$\\
\bottomrule
\end{longtable*}

\subsection{Abbreviations and Acronyms}

\renewcommand{\arraystretch}{1.2}
\begin{tabular}{l l} 
  \toprule		
  \textbf{symbol} & \textbf{description}\\
  \midrule 
  A & Assumption\\
  $\Cz$ & Complex numbers\\
  DD & Data Definition\\
  GD & General Definition\\
  GS & Goal Statement\\
  IM & Instance Model\\
  \ivp & Initial Value Problem\\
  LC & Likely Change\\
  $\Iz$ & The non-negative integers\\
  \ode & Ordinary Differential Equation\\
  $\Rz$ & Real numbers\\
  R & Requirement\\
  \progname{f} & Radius of Convergence software developed for this project\\
  SRS & Software Requirements Specification\\
  T & Theoretical Model\\
  TC & Taylor coefficient\\
  TS & Taylor series\\
  TLA & Top Line Analysis\\
  3TA & Three Term Analysis\\
  6TA & Six Term Analysis\\
  \bottomrule
\end{tabular}\\

\newpage

\pagenumbering{arabic}

%CHAPTER
\section{Introduction} \label{sc:introduction}

Given a sequence $\Setbg{c_n}$ of complex numbers, the series
\EQ
{
  \label{eq:power-series}
  \sum_{n=0}^{\infty} c_n (z-z_0)^n
}
is called a {\it power series}. The number $c_n  \in \Cz$ is the $n^\text{th}$ coefficient in the power series.
The symbol $z^n$ denotes the $n^\text{th}$ power of the complex number $z$. This power series is {\it centered}
at $z_0 \in \Cz$.

In general, a power series will converge or diverge, depending on the magnitude of $z-z_0$.
With every power series, there is associated a circle of convergence such that
\eqref{power-series} converges if $z$ is in the interior of the circle of convergence or
diverges if $z$ is in the exterior of the circle of convergence. The convergence/divergence
behavior of \eqref{power-series} on the circle of convergence can not be described so simply.
By convention, the entire complex plane is the interior of a circle of infinite radius, and a
point is the interior of a circle of zero radius.

This project is concerned with estimating the radius $R_c$ of the circle of convergence.

\subsection{Purpose of Document}

The purpose of this document is to facilitate communication between the stakeholders and developers
 during the software development of project \progname{f}
 %design, software development and testing, verification and validation, and documentation of \progname{f}
 by communicating and reflecting its software requirements.
The scientific and business problem \progname{f} solves is described in
Section~\ref{Sec_pd}, the \Quote{Problem~Description}. 

\subsection{Scope of Requirements}\label{sc:scope}

In the late 1800's, several authors resolved problems
concerning the characterization and analysis of singularities for power series.
\cite{chang1982} discuss this history and their approach to computing $R_c$.

\begin{quote}
Our approach to series analysis was motivated by the observation that series
for solutions to \ode{s} follow a few very definite patterns which are characterized
by the location of primary singularities. In general, the coefficients of a power
series follow no patterns, so few theorems about truncated series can be proved.
However, series which are real-valued on the real axis can have poles, logarithmic
branch points, and essential singularities only on the real axis or in conjugate pairs.
Further, the effects of all secondary singularities disappear if sufficiently
  long series are used.  \citep[p.~122]{chang1982}
\end{quote}

A primary singularity of \eqref{power-series} is the closest singularity to the series
expansion point in the complex plane. All other singularities are secondary singularities.

\cite{chang1982} proposed a method of four parts to estimate the radius $R_c$ of the circle of convergence
as well as the order and location of primary singularities. The top-hump analysis applies to
the power series of entire functions. The 3TA analysis applies to the power series of
functions exhibiting a single primary singularity. The 6TA applies to the
power series of functions exhibiting a conjugate pair of primary singularities.
Whenever these three analysis fail to resolve the $R_c$, singularity order, and
singularity location parameters for the series, the \cite{chang1982} method does
a top-line analysis. Each \cite{chang1982} sub-method works by fitting $R_c$, singularity order, and
singularity location parameters of a known model to the given sequence.

Top-line analysis always applies to power series. It resolves situations where secondary singularities are
less distinguishable from primary singularities. However it is less accurate, but it does
have a convergence analysis \citep{chang1982}.

The scope of this \progname{f} project includes three term analysis of primary real poles,
six term analysis of primary pair of complex conjugate poles, and top line analysis.
\progname{f} does not yet include an analysis for essential singularities.

\subsection{Characteristics of Intended Reader} \label{sec_IntendedReader}

This document assumes the intended reader has familiarity with basic real analysis, complex analysis,
and Taylor arithmetic.
Courses which contribute to background knowledge may be titled Ordinary Differential Equations and
Linear Algebra (undergrad), Introduction to Real Analysis (undergrad),
Multivariate Calculus (undergrad), Functional Analysis (Graduate), Real Analysis (Graduate),
and Complex Analysis (Graduate).
Sequences and power series as well as the ratio and root tests will be discussed
in this document. However, our exposition will only cover the concepts needed for our purposes. For proofs
and for a complete exposition of all background materials, the interested reader should consult a beginning
level graduate text such as \cite{rudin1976}. For a brief but sufficient introduction to Taylor arithmetic,
consult \cite{TADIFF}.

\subsection{Organization of Document}

This document is built on the template recommendations in \citet{SmithAndLai2005, SmithEtAl2007} that
 seeks to standardize communication tools for software development. The suggested order for reading
 this SRS document is: Goal Statement (\SSCref{goal-statements}), Instance Models (\SSCref{instance-models}),
 Requirements (\SCref{requirements}), Introduction (\SCref{introduction}), and
 Specific System Description (\SCref{specific-system-description}).

%CHAPTER
\section{General System Description}

This section provides general information about the system.  It identifies the
interfaces between the system and its environment, describes the user
characteristics, and lists the system constraints.  \plt{This text can likely be
  borrowed verbatim.}

\subsection{System Context}\label{ssc:systemcontext}

The following figure depicts a system context view of \progname{f}. This
context appears, for example, in the numerical solution of ordinary differential and
differential algebraic equations.

\begin{figure}[h!]
\begin{center}
 \includegraphics[width=0.95\textwidth]{roc-system-context}
\caption{System Context}
\label{fg:systemcontext} 
\end{center}
\end{figure}

After generating a real-valued Taylor series (TS) approximate solution of order $p$ to the
ordinary differential equation (\ode) initial-value problem (\ivp) 
\EQ
{
  \label{eq:introduction-mathematical-odeivp}
  y'\parn{t} = f\pbg{t,y\parn{t}},
  \quad
  y\parn{t_0} = y_0 \in \mathcal D \subset \Rz^d,
  \quad
  t \in \iode \subset \Rz,
}
the TS methods defined in
\cite{jorba2005software},
\cite{bergsma2016application},
and \cite{chang1982} explicitly require an estimate for the TS radius of convergence.

At each $\parn{t_n,x_n}$, $n \geq 0$, the TS method for the numerical solution
of \eqref{introduction-mathematical-odeivp} computes Taylor coefficients (TCs)
$\tc{\xn}{i}$ at $t_{n}$ to construct the TS approximate solution
\begin{align}
  \label{eq:tss}
  \Tp(t) = \xn + \sum_{i=1}^{p}\tc{\xn}{i}(t-\tn)^{i} \quad\mbox{on}\quad \lode.
\end{align}
Analysis of \EQref{tss} for its radius of convergence provides a practical system
context for \progname{f} as depicted in \FGref{systemcontext}.
In this system context, developers like \cite{bergsma2016application},
\cite{jorba2005software}, or \cite{chang1982} of a TS method seek the accuracy
assurance from knowing the domain $\lode$ is in the circle of convergence.

\subsection{User Characteristics} \label{SecUserCharacteristics}
\label{sc:SecUserCharacteristics}

One intended user of \progname{f} is a user of \maple or \matlab. While
neither \matlab nor \maple currently implements an estimate for $R_c$,
presumably the companies that develop \matlab and \maple will want
to provide such a facility when robust, reliable, and accurate computational
tools are available.
Users of \progname{f} would be a calculus student, a user of \maple or \matlab, and a
developer of a Taylor series method as in \SSCref{systemcontext}.

\subsection{System Constraints}

The method developed in this project is expected to be independent of system constraints. However
most TS methods are developed in \cpp or \fortran, the goto languages of scientific computing.
Certainly a scripting language would not be sufficient for large systems.

%CHAPTER
\section{Specific System Description} \label{sc:specific-system-description}

This section first presents the problem description, which gives a high-level
view of the problem to be solved.  This is followed by the solution characteristics
specification, which presents the assumptions, theories, definitions and finally
the instance models.

\subsection{Problem Description} \label{Sec_pd}

\progname{f} is intended to estimate the radius of the circle of convergence of a power series.

\subsubsection{Terminology and  Definitions}\label{ssc:terminology-definitions}

A {\it sequence} is a function $f$ whose domain is the non-negative integers $\Iz$
and range is in $E$, that is, a sequence is the mapping $n \in \Iz \mapsto f(n) = c_n \in E$. It is customary
to denote the sequence $f$ by the symbol $\Setbg{c_n}$ or by $c_0, c_1, c_2, \ldots$.
The values of $f$, that is, the elements $c_n$ are called the {\it terms} of the sequence.
If $A$ is a subset of $E$ and if $c_n \in A$ for all $n \in \Iz$, the $\Setbg{c_n}$ is said to be
a {\it sequence in $A$}. The terms of a sequence need not be distinct. Typically, $E$ is
the complex numbers $\Cz$ or the real numbers $\Rz$.

Given a sequence $\Setbg{c_n}$, we use the notation
\EQ
{
  \sum_{n=p}^q c_n \qquad\text{with}\qquad p \leq q
}
to denote the sum $c_p + c_{p+1} + \cdots + c_q$. With the sequence $\Setbg{c_n}$, we associate a sequence
$\Setbg{s_n}$, where
\EQ
{
  \label{eq:partial-sum}
  s_n \defeq \sum_{k=0}^n c_k.
}
For the sequence $\Setbg{s_n}$, we may use the symbolic expression $c_0 + c_1 + c_2 + \ldots$ or
\EQ
{
  \label{eq:series-symbol}
  \sum_{n=0}^{\infty} c_n.
}

The symbol \eqref{series-symbol} is called an {\it infinite series} or just {\it series}. The
terms $s_n$ are called {\it partial sums} of the series, they are just numbers.
If $\Setbg{s_n} \rightarrow s$ as $n \rightarrow \infty$, then we say $\Setbg{s_n}$ converges to $s$,
the series converges, and write
\EQ
{
  \label{eq:series-limit}
  \sum_{n=0}^{\infty} c_n = s.
}
The number $s$ is the limit of a sequence of sums called the {\it sum of the series}. If $\Setbg{s_n}$
diverges, then the series is said to diverge.

Given a sequence $\Setbg{e_n}$, consider a sequence $\Setbg{n_k}$ of non-negative integers such that
$n_0 < n_1 < n_2 < \cdots$. Then the sequence $\Setbg{e_{n_k}}$ is called a {\it subsequence} of
$\Setbg{e_n}$. If $\Setbg{e_{n_k}}$ converges, its limit is called a {\it subsequential limit} of $\Setbg{e_n}$.

Let $\Setbg{s_n}$ be a sequence of real numbers. Let $E$ be the set of numbers $x$ in the extended real number
system such that $s_{n_k} \rightarrow x$ for some subsequence $\Setbg{s_{n_k}}$.
This set $E$ contains all subsequential limits plus possibly $+\infty$ and $-\infty$. Define $s^* \defeq \sup E$ and
$s_* \defeq \inf E$. The numbers $s^*$ and $s_*$ are called the {\it upper limit} and {\it lower limit}
of $\Setbg{s_n}$. We use the notation
\EQ
{
  \nliminf{n} s_n = s_* \qquad\text{and}\qquad \nlimsup{n} s_n = s^*.
}

\subsubsection{Physical System Description} \label{sec_phySystDescrip}

This subsection doesn't apply to \progname{f}.

\subsubsection{Goal Statements}\label{ssc:goal-statements}

\Ni Given a tolerance \tol, a truncated finite sequence $\Setbg{c_n}$, and the applied scale
to the coefficients $c_n$, the goals of this project are:

\begin{itemize}
\item[GS\refstepcounter{goalnum}\thegoalnum \label{G_3ta}:] Implement \cite{chang1982} three term analysis
to estimate $R_c$, the order of singularity, the modelling error, and the truncation error committed
by truncating the sequence at $N$ for a primary real pole. This appears to be four goals. However,
the same setup and computation leads to each quantity. Separating these quantities into four goals
would lead to a verbose, redundant document.

\item[GS\refstepcounter{goalnum}\thegoalnum \label{G_6ta}:] Implement \cite{chang1982} six term analysis
to estimate $R_c$, the order of singularity, the modelling error, and the truncation error committed
by truncating the sequence at $N$ for a pair of primary complex conjugate poles. This appears to be
four goals. However, the same setup and computation leads to each quantity. Separating these quantities
into four goals would lead to a verbose, redundant document.

\item[GS\refstepcounter{goalnum}\thegoalnum \label{G_topline}:] Implement \cite{chang1982} top line analysis
to estimate $R_c$ and the modelling error for a mix of primary and secondary poles,
logarithmic branch points, and essential singularities.
\end{itemize}

\subsection{Solution Characteristics Specification}

This section characterizes the attributes of an acceptable solution.
Both analysts and stakeholders should agree on these attributes
so that the solution can be accepted when the project is complete.

\subsubsection{Assumptions} \label{sec_assumpt}

Given a tolerance \tol, consider the sequence $\Setbg{c_n}$ and its power series
$\sum_{n=0}^{\infty} c_n (z-z_0)^n$
under the following assumptions:

\begin{itemize}

\item[A\refstepcounter{assumpnum}\theassumpnum \label{as:n}:]
  We know an integer $N$ such that, for all $m \geq n \geq N$,
  $| \sum_{k=n}^{m} c_k | < \tol$.

\item[A\refstepcounter{assumpnum}\theassumpnum \label{as:approximate}:]
  The software \progname{f} will estimate the radius of convergence from a finite number of terms
  in the power series. It will not compute $R_c$ exactly. 

\item[A\refstepcounter{assumpnum}\theassumpnum \label{as:real}:]
  The sequence $\Setbg{c_n}$ is a subset of $\Rz$.

\item[A\refstepcounter{assumpnum}\theassumpnum \label{as:threeterm}:]
  The scope of three term analysis is to compute the radius of convergence of
  a real power series at a point $z_0 \in \Rz$ that has a primary singularity.

\item[A\refstepcounter{assumpnum}\theassumpnum \label{as:sixterm}:]
  The scope of six term analysis is to compute the radius of convergence of
  a real power series at a point $z_0 \in \Rz$ that has a complex conjugate pair of
  primary singularity.

\item[A\refstepcounter{assumpnum}\theassumpnum \label{as:topline}:]
  The scope of top line analysis is to compute the radius of convergence of
  a real power series at a point $z_0 \in \Rz$ that has a secondary singularity.

\end{itemize}

\subsubsection{Theoretical Models}\label{sec_theoretical}
\label{ssc:TM}

Applying the terminology and definitions from \SSCref{terminology-definitions}, this section records
theorems required to identify a convergent/divergent series.

Consider the sequence $\Setbg{c_n}$ and its power series $\sum_{n=0}^{\infty} c_n (z-z_0)^n$. The following
TM is used to show that the coefficients in the terms of a series tend to zero as the index of the term
tends to infinity.

\noindent
\begin{minipage}{\textwidth}
\renewcommand*{\arraystretch}{1.5}
\begin{tabular}{| p{\colAwidth} | p{\colBwidth}|}
  \hline
  \rowcolor[gray]{0.9}
  Number& T\refstepcounter{theorynum}\thetheorynum \label{TM-series-cauchy-condition}\\
  \hline
  Label&\bf Cauchy convergence condition\\
  \hline
  Theorem& A series $\sum_{n=0}^{\infty} c_n$ converges if an only if, for every $\epsilon > 0$,
  there is an integer $N$ such that $| \sum_{k=n}^{m} c_k | < \epsilon$ whenever $m \geq n \geq N$.\\ 
  \hline
  Description & Tools to identify when a series converges.\\
  \hline
  Source & Theorem 3.22 \citep[p.~59]{rudin1976}\\
  \hline
  Ref.\ By & \iref{IM-rc}\\
  \hline
\end{tabular}
\end{minipage}\\

~\newline

\noindent
\begin{minipage}{\textwidth}
\renewcommand*{\arraystretch}{1.5}
\begin{tabular}{| p{\colAwidth} | p{\colBwidth}|}
  \hline
  \rowcolor[gray]{0.9}
  Number& T\refstepcounter{theorynum}\thetheorynum \label{TM-convergence-of-sequence}\\
  \hline
  Label&\bf Convergence of sequence\\
  \hline
  Theorem& If series $\sum_{n=0}^{\infty} c_n$ converges, then $\lim_{n \rightarrow \infty} c_n = 0$.\\
  \hline
  Description & If a series converges, then its terms converge to zero.\\
  \hline
  Source & Theorem 3.28 \citep[p.~60]{rudin1976}\\
  \hline
  Ref.\ By & \iref{IM-rc}\\
  \hline
\end{tabular}
\end{minipage}\\

~\newline

\noindent
\begin{minipage}{\textwidth}
\renewcommand*{\arraystretch}{1.5}
\begin{tabular}{| p{\colAwidth} | p{\colBwidth}|}
  \hline
  \rowcolor[gray]{0.9}
  Number& T\refstepcounter{theorynum}\thetheorynum \label{TM-root-test}\\
  \hline
  Label&\bf Root test\\
  \hline
  Theorem&
  \begin{minipage}[t]{0.8\textwidth}  
    Given a series $\sum_{n=0}^{\infty} c_n$. Set $\alpha \defeq \nlimsup{n} \proot{n}{|c_n|}$.
  Then
  \begin{itemize}
    \item[(a)] if $\alpha < 1$, then $\sum_{n=0}^{\infty} c_n$ converges;
    \item[(b)] if $\alpha > 1$, then $\sum_{n=0}^{\infty} c_n$ diverges;
    \item[(c)] if $\alpha = 1$, then this test gives no information.
  \end{itemize}
  \end{minipage}\\
  \hline
  Description & Tools to identify when a series converges/diverges.\\
  \hline
  Source & Theorem 3.33 \citep[p.~65]{rudin1976}\\
  \hline
  Ref.\ By & \dref{GD-rc} and \iref{IM-rc}\\
  \hline
\end{tabular}
\end{minipage}\\

~\newline

\noindent
\begin{minipage}{\textwidth}
\renewcommand*{\arraystretch}{1.5}
\begin{tabular}{| p{\colAwidth} | p{\colBwidth}|}
  \hline
  \rowcolor[gray]{0.9}
  Number& T\refstepcounter{theorynum}\thetheorynum \label{TM-ratio-test}\\
  \hline
  Label&\bf Ratio test\\
  \hline
  Theorem&
  \begin{minipage}[t]{0.8\textwidth} 
  The series $\sum_{n=0}^{\infty} c_n$
  \begin{itemize}
    \item[(a)] converges if $\nlimsup{n} \lvert \tfrac{c_{n+1}}{c_n} \rvert < 1$,
    \item[(b)] diverges if $\lvert \tfrac{c_{n+1}}{c_n} \rvert \geq 1$ for $n \geq N$, where $N$ is some fixed integer.
  \end{itemize}
  \end{minipage}\\
  \hline
  Description & Tools to identify when a series converges/diverges.\\
  \hline
  Source & Theorem 3.34 \citep[p.~66]{rudin1976}\\
  \hline
  Ref.\ By & \iref{IM-rc}\\
  \hline
\end{tabular}
\end{minipage}\\

~\newline
The ratio test is often easier to apply than the root test. However, the root test resolves more application
than the ratio test. Both the ratio test and the root test deduce divergence from the statement in
Theoretical Model \ref{TM-convergence-of-sequence}, if a series converges, then its terms converge to zero.
~\newline

\noindent
\begin{minipage}{\textwidth}
\renewcommand*{\arraystretch}{1.5}
\begin{tabular}{| p{\colAwidth} | p{\colBwidth}|}
  \hline
  \rowcolor[gray]{0.9}
  Number& T\refstepcounter{theorynum}\thetheorynum \label{TM-comparing-ratio-and-root}\\
  \hline
  Label&\bf Comparing the Ratio test and the Root test\\
  \hline
  Theorem& 
  \begin{minipage}[t]{0.8\textwidth} 
  For any sequence $\Setbg{c_n}$ of positive (real) numbers,
  \begin{itemize}
    \item[(a)] $\nliminf{n} \tfrac{c_{n+1}}{c_n} \leq \nliminf{n}\proot{n}{c_n}$,
    \item[(b)] $\nlimsup{n}\proot{n}{c_n} \leq \nlimsup{n} \tfrac{c_{n+1}}{c_n}$
  \end{itemize}
  \end{minipage}\\
  \hline
  Description & If the ratio test converges,
  then the root test converges and if the root test is inconclusive, then the ratio test is inconclusive.
  Whenever the limit exists and it is unique, then there is equality in (a) and (b) and
  $\lim_{n \rightarrow \infty} \tfrac{c_{n+1}}{c_n} = \lim_{n \rightarrow \infty} \proot{n}{c_n}$.
  \\
  \hline
  Source & Theorem 3.37 \citep[p.~68]{rudin1976}\\
  \hline
  Ref.\ By & \iref{IM-rc}\\
  \hline
\end{tabular}
\end{minipage}\\

~\newline
The characterization and analysis of singularities of analytic functions
by its Laurent series was known by the 1870s \citep[p.~122]{chang1982}.
The following theoretical model characterizes singularities
that a power series with real coefficients can exhibit.
~\newline

\noindent
\begin{minipage}{\textwidth}
\renewcommand*{\arraystretch}{1.5}
\begin{tabular}{| p{\colAwidth} | p{\colBwidth}|}
  \hline
  \rowcolor[gray]{0.9}
  Number& T\refstepcounter{theorynum}\thetheorynum \label{TM-primary-poles}\\
  \hline
  Label&\bf Series with primary poles\\
  \hline
  Theorem& 
  \begin{minipage}[t]{0.8\textwidth} 
  For any sequence $\Setbg{c_n}$ of real numbers,
  its power series $\sum_{n=0}^{\infty} c_n (z-z_0)^n$ when $z_0 \in \Rz$ can have poles,
  logarithmic branch points, and essential singularities only on the real axis or in conjugate pairs.
  Further, the effects of all secondary singularities disappear if sufficiently long series are used.\\
  \end{minipage}\\
  \hline
  Description & Characterization of singularities of power series with real coefficients.
  \\
  \hline
  Source & \citep[p.~122]{chang1982}\\
  \hline
  Ref.\ By & \iref{IM-3T}, \iref{IM-3TA}, \iref{IM-6T}, \iref{IM-6TA}, and \iref{IM-rc}\\
  \hline
\end{tabular}
\end{minipage}\\

\subsubsection{General Definitions}\label{sec_gendef}

The proofs of the theorem in this section apply the terminology and definitions
from \SSCref{terminology-definitions} as well as the Theoretical models from \SSCref{TM}.

The radius of the circle of convergence is defined in the next General Definition, a theorem
that enables us to justify and construct the IM for TLA so that \progname{f}
will estimate $R_c$.
~\newline

\noindent
\begin{minipage}{\textwidth}
\renewcommand*{\arraystretch}{1.5}
\begin{tabular}{| p{\colAwidth} | p{\colBwidth}|}
  \hline
  \rowcolor[gray]{0.9}
  Number& GD\refstepcounter{defnum}\thedefnum \label{GD-rc}\\
  \hline
  Label&\bf Define the radius of the circle of convergence\\
  \hline
  Theorem& Given any sequence $\Setbg{c_n}$, construct the power series
  $\sum_{n=0}^{\infty} c_n (z-z_0)^n$. Set $\alpha \defeq \nlimsup{n} \proot{n}{|c_n|}$ and $R_c \defeq 1/\alpha$.
  Then $\sum_{n=0}^{\infty} c_n (z-z_0)^n$ converges whenever $|z - z_0| < R_c$.\\
  \hline
  Description & This General Definition defines $R_c$, the radius of convergence of the power series.
  By our convention stated \SSCref{terminology-definitions}, $\alpha = 0$ implies $R_c = +\infty$ and
  $\alpha = +\infty$ implies $R_c = 0$.\\
  \hline
  Source & Theorem 3.39 \citep[p.~69]{rudin1976}\\
  \hline
  Ref.\ By & \iref{IM-rc}\\
  \hline
\end{tabular}
\end{minipage}\\

~\newline
We need to relate the root test to the ratio test to obtain our IM. It is instructive to
understand the role of the root test in the proof of \dref{GD-rc}.

\subsubsection*{Inside \dref{GD-rc}}

Given any sequence $\Setbg{c_n}$, construct the power series
$\sum_{n=0}^{\infty} c_n (z-z_0)^n$. Set $a_n = c_n (z - z_0)^n$, and apply the root test \tref{TM-root-test}
to the series $\sum_{n=0}^{\infty} a_n$.
\EQ
{
  \label{eq:rc-definition}
  \nlimsup{n} \proot{n}{|a_n|} = |z-z_0| \nlimsup{n} \proot{n}{|c_n|} \defeq \tfrac{|z-z_0|}{R_c}.
}
Obtain from the root test that
\begin{itemize}
  \item[(a)] if $|z-z_0| < R_c$, then the power series converges;
  \item[(b)] if $|z-z_0| > R_c$, then the power series diverges;
  \item[(c)] if $|z-z_0| = R_c$, then this test gives no information.
\end{itemize}

~\newline
The next section presents a Data Definition on order of singularity.

\subsubsection{Data Definitions}\label{sec_datadef}

We quoted from \cite{chang1982} in \SCref{scope}, the scope section, that,
in general, the coefficients of a power series follow no patterns, so few theorems about truncated
series can be proved. However, \tref{TM-primary-poles} asserts that
series which are real-valued on the real axis can have poles,
logarithmic branch points, and essential singularities only on the real axis or in conjugate pairs.
The sequence $\Setbg{c_n}$ is a subset of $\Rz$ under \ASref{real}.
The following DD concerns a procedure to approximate the order of singularity in TLA for
series which are real-valued on the real axis.
~\newline

\noindent
\begin{minipage}{\textwidth}
\renewcommand*{\arraystretch}{1.5}
\begin{tabular}{| p{\colAwidth} | p{\colBwidth}|}
\hline
\rowcolor[gray]{0.9}
Number& DD\refstepcounter{datadefnum}\thedatadefnum \label{DD-order-of-singularity}\\
\hline
Label& \bf Order of singularity\\
\hline
Symbol &$\mu$\\
\hline
  Conditions &
  Assume \ASref{real}.
Further assume the real coefficients $\Setbg{c_n} \subset \Rz$ of the power series
$\sum_{n=0}^{\infty} c_n (z-z_0)^n$ are obtained as a TS solution of an \ode and
consider finding the order $\mu$ of the singularity from the graph
of $\log_{10} | c_n |$ versus $n$.\\
\hline
  Observations &
  \begin{minipage}[t]{0.8\textwidth} 
The order $\mu$ is increased or decreased by term-by-term differentiation or
integration, respectively. The upper envelope of the graph of
$\log_{10} | c_n |$ versus $n$ will follow the following patterns:
\begin{itemize}
  \item If the order of the primary singularity, the closest singularity to $z_0$, is $\mu = 1$,
    then the slope is $\log_{10} |z - z_0|/R_c$.
    
  \item If the order of the primary singularity $\mu \neq 1$, then the slope converges to
    $\log_{10} |z - z_0|/R_c$ at a rate proportional to $1/n$.

  \item If the order of the primary singularity $\mu \neq 1$, then the upper envelope is not linear.
    For orders $\mu > 1$, the graph opens downward. The concavity approaches zero as
    $1/n^2$ as $n \rightarrow \infty$. For orders $\mu < 1$, the graph is concave up which means
    the slope underestimates $\log_{10} |z - z_0|/R_c$, and $R_c$ is overestimated.
\end{itemize}
  \end{minipage}\\
  \hline
Description &\\
  \hline
  Sources& \cite{chang1982}\\
  \hline
  Ref.\ By & \iref{IM-order-of-singularity}\\
  \hline
\end{tabular}
\end{minipage}\\

Our analysis requires the following DD, a basic formula of Taylor arithmetic.
~\newline

\noindent
\begin{minipage}{\textwidth}
\renewcommand*{\arraystretch}{1.5}
\begin{tabular}{| p{\colAwidth} | p{\colBwidth}|}
\hline
\rowcolor[gray]{0.9}
Number& DD\refstepcounter{datadefnum}\thedatadefnum \label{DD-taylor-arithmetic-constant-power}\\
\hline
Label& \bf Taylor arithmetic formula: analytic function to a constant power\\
\hline
  Symbol & $\tc{u^a}{k}$\\
\hline
  Conditions & Let $z \in \mathcal D \subset \Rz \mapsto u(z) \in \Rz$ be an analytic function.
  Denote by $\tc{u}{k}$ the $k^{\text{th}}$ TC of $u$ evaluated at some real $z_0 \in \mathcal D$.
  Assume $u(z_0) \neq 0$. Let $a$ be a real constant.\\
\hline
  Observations &
  \begin{minipage}[t]{0.8\textwidth} 
    Then the $k^{\text{th}}$ TC of $\tc{u^a}{k}$ is given by the Taylor arithmetic formula
    \begin{equation}
      \tc{u^a}{k} = \frac{1}{k \tc{u}{0}} \sum_{j=0}^{k-1} (a(k-j) - j) \tc{u^a}{j} \tc{u}{k-j},
      \qquad \text{for} k \geq 1.
    \end{equation}
  \end{minipage}\\
  \hline
  Description & TCs of an analytic function to a constant power\\
  \hline
  Sources& \cite{TADIFF}\\
  \hline
    Ref.\ By & \iref{IM-3T} and \iref{IM-6T}\\
  \hline
\end{tabular}
\end{minipage}\\

~\newline
The next section derives an IM to approximate $R_c$.

\subsubsection{Instance Models} \label{sec_instance}    
\label{ssc:instance-models}

This section transforms the problem defined in Section~\ref{Sec_pd} into 
one which can be translated into software. We will define finite sequences
and series to replace the infinite counterparts
identified in Sections~\ref{sec_theoretical} and~\ref{sec_gendef}.

Given real constants $a$ and $s>0$, real expansion point $z_0 = 0$, and scaling $h$,
\cite{chang1982} choose the real valued function
\EQ
{
  \label{eq:model-real-pole}
  v(z) \defeq ( a - z )^{-s}
}
to model one primary pole or logarithmic branch point. The model is
relevant because, for any fixed but arbitrary $z_0$, any analytic
function $f$ at $z_0$ with only one primary pole or logarithmic branch point
has the form $f(z) = C(v(z))$, where $C$ is an analytic function near
$z_0$ \citep{chang1982}.

~\newline
\noindent
\begin{minipage}{\textwidth}
\renewcommand*{\arraystretch}{1.5}
\begin{tabular}{| p{\colAwidth} | p{\colBwidth}|}
  \hline
  \rowcolor[gray]{0.9}
  Number& IM\refstepcounter{instnum}\theinstnum \label{IM-3T}\\
  \hline
  Label& TCs for one primary real pole or logarithmic branch point\\
  \hline
  Theorem& 
  \begin{minipage}[t]{0.8\textwidth} 
    Given real constants $a$ and $s>0$, real expansion point $z_0 = 0$, and scaling $h$,
    the TCs of \EQref{model-real-pole} at $z_0 = 0$ are
    \begin{equation}
      k \tc{v}{k} = \tc{v}{k-1} (k + s - 1) \frac{h}{z_0}, \qquad \text{for} \quad k \geq 1.
    \end{equation}
  \end{minipage}\\
  \hline
  Description & TCs of \EQref{model-real-pole}\\
  \hline
  Sources& \cite{chang1982}\\
  \hline
    Ref.\ By & \iref{IM-3TA}\\
  \hline
\end{tabular}
\end{minipage}\\

~\newline
We now justify \iref{IM-3T}.

\subsubsection*{Inside \iref{IM-3T}}
Given real constants $a$ and $s>0$, real expansion point $z_0$, and scaling $h$,
\EQref{model-real-pole} is a pole or a logarithmic branch point, which are two possibilities
of the four total enumerated in \tref{TM-primary-poles}.

Compute the TS for \EQref{model-real-pole} at $z_0 = 0$. $R_c$ is the distance to primary singularity.
We employ the Taylor arithmetic formula for
analytic function to a constant power provided in \ddref{DD-taylor-arithmetic-constant-power}.
Set $u = a - z$. Then $\tc{u}{0} = z_0$, $\tc{u}{1} = -1$, and $\tc{u}{k} = 0$ for $k>1$.
Compute $j = k \Rightarrow k-j = 0$ and $j = k-1 \Rightarrow k-j = 1$. A scaled $R_c$ is
related to the unscaled $\widetilde R_c$ by $h \widetilde R_c = R_c$. As
a matter of notation, $\widetilde R_c = z_0$. Apply \ddref{DD-taylor-arithmetic-constant-power}
to this problem and simplify to observe the statement of \iref{IM-3T}.

~\newline
\noindent
\begin{minipage}{\textwidth}
\renewcommand*{\arraystretch}{1.5}
\begin{tabular}{| p{\colAwidth} | p{\colBwidth}|}
  \hline
  \rowcolor[gray]{0.9}
  Number& IM\refstepcounter{instnum}\theinstnum \label{IM-3TA}\\
  \hline
  Label& 3TA: Model one primary real pole or logarithmic branch point\\
  \hline
  Theorem& 
  \begin{minipage}[t]{0.8\textwidth} 
    Given tolerance \tol, real constants $a$ and $s>0$, a real expansion point $z_0$,
    scaling $h$, $N > \nuses > \minterms > 0$, and real coefficients $\Setbg{c_n} \subset \Rz$
    of the power series $\sum_{n=0}^{\infty} c_n (z-z_0)^n$, set
    $W(i, 1) = (k(i)- 1) c_{k(i)}$, $W(i, 2) = c_{k(i)}$, and
    $b(i) = k(i) c_{k(i) + 1}$ for $k(i)=N-\nuses, \ldots, N$ and $i=1, \ldots, \nuses$
    to obtain the linear system $W \beta = b$ where $\beta(1) = h/R_c \defeq h/z_0$
    and $\beta(2) = s \beta(1)$. Let $\beta$ solve
    $\min_{\beta \in \Rz^2} \|\beta\|_2^2$ such that $W \beta = b$.
    Then the least squares best fit of the data $\Setbg{c_n}$ to the model problem \eqref{model-real-pole}
    is $R_c = h/\beta(1)$ and $\mu = \beta(2)/\beta(1)$ .
  \end{minipage}\\
  \hline
  Description & Use 3TA to compute $R_c$ and $\mu$ and approximate truncation error\\
  \hline
  Sources& The full details are my own contribution. However the ideas are inspired by conversations
  with G. Corliss and N. Nedialkov.\\
  \hline
    Ref.\ By & Final product\\
  \hline
\end{tabular}
\end{minipage}\\

~\newline
We now justify \iref{IM-3TA}.

\subsubsection*{Inside \iref{IM-3TA}}

Given tolerance \tol,
let \ASref{n} and \ASref{real} hold. \ASref{n} provides the parameter $N$ such that, for all
$m \geq n \geq N$, $| \sum_{k=n}^{m} c_k | < \tol$.
Given real constants $a$ and $s>0$, a real expansion point $z_0$,
scaling $h$, $N > \nuses > \minterms > 0$, and real coefficients $\Setbg{c_n} \subset \Rz$
of the power series $\sum_{n=0}^{\infty} c_n (z-z_0)^n$, we formulate the optimization problem.

We use optimal least squares fitting to fit the real
coefficients $\Setbg{c_{\nuses}, \ldots, c_N }$ from any given real sequence $\Setbg{c_n}$
to the sequence $\Setbg{v_n}$ of \iref{IM-3T}.

We rearrange the \iref{IM-3T} system
\EQ
{
  \label{eq:3TA}
  k \tc{v}{k} = \tc{v}{k-1} (k + s - 1) \frac{h}{z_0}, \qquad \text{for} \quad k \geq 1.
}
to
\EQ
{
  k \tc{v}{k} = \tc{v}{k-1} (k - 1) \frac{h}{z_0} + \tc{v}{k-1} s \frac{h}{z_0},
  \qquad \text{for} \quad k \geq 1.
}
Set
$W(i, 1) = (k(i)- 1) c_{k(i)}$, $W(i, 2) = c_{k(i)}$, and
$b(i) = k(i) c_{k(i) + 1}$ for $k(i)=N-\nuses, \ldots, N$ and $i=1, \ldots, \nuses$
to obtain the linear system $W \beta = b$ where $\beta(1) = h/R_c \defeq h/z_0$
and $\beta(2) = s \beta(1)$.

At the optimal solution, $R_c = h/\beta(1)$ and $\mu = \beta(2)/\beta(1)$.

Suppose $R_c$ and $\mu$ computed as in \iref{IM-3TA} have a small fitting error.
The quality for small is not yet defined. Then $R_c$ and $\mu$ satisfy
\ASref{approximate} and \ASref{threeterm}. Moreover using
$\tc{v}{k}$ as $c_k$ in \ASref{n}, we have an estimate for the truncation error.

Given real constants $a$, $b$, and $s>0$, real expansion point $z_0 = 0$, and scaling $h$,
\cite{chang1982} choose the real valued function
\EQ
{
  \label{eq:model-complex-pair-pole}
  w(z) \defeq ( 1/2 z^2 - 2 b z + a^2 )^{-s}
}
to model one complex conjugate pair of primary pole.
Let $\cos \theta \defeq b/a$.

~\newline
\noindent
\begin{minipage}{\textwidth}
\renewcommand*{\arraystretch}{1.5}
\begin{tabular}{| p{\colAwidth} | p{\colBwidth}|}
  \hline
  \rowcolor[gray]{0.9}
  Number& IM\refstepcounter{instnum}\theinstnum \label{IM-6T}\\
  \hline
  Label& TCs for one complex conjugate pair of primary pole\\
  \hline
  Theorem& 
  \begin{minipage}[t]{0.8\textwidth} 
    Given real constants $a$, $b$, and $s>0$, real expansion point $z_0 = 0$, and scaling $h$,
    the TCs of \EQref{model-complex-pair-pole} at $z_0 = 0$ are
    \begin{equation}
      k \tc{w}{k} = 2 \tc{w}{k-1} (k +   s - 1) \frac{h}{z_0} \cos \theta
                    -\tc{w}{k-2} (k + 2 s - 2) \left(\frac{h}{z_0}\right)^2,
                   \quad \text{for} \quad k \geq 1.
    \end{equation}
  \end{minipage}\\
  \hline
  Description & TCs of \EQref{model-complex-pair-pole}\\
  \hline
  Sources& \cite{chang1982}\\
  \hline
    Ref.\ By & \iref{IM-6TA}\\
  \hline
\end{tabular}
\end{minipage}\\

~\newline
We now justify \iref{IM-6T}.

\subsubsection*{Inside \iref{IM-6T}}
Given real constants $a$, $b$, and $s>0$, real expansion point $z_0 = 0$, and scaling $h$,
\EQref{model-complex-pair-pole} is a complex conjugate pair of pole, which is one more
additional possibility of the four total enumerated in \tref{TM-primary-poles},
distinct from the two covered possibilities discussed in \iref{IM-3T}.

Compute the TS for \EQref{model-complex-pair-pole} at $z_0 = 0$,
$R_c$ is the distance to primary singularity,
using the Taylor arithmetic formula for
analytic function to a constant power provided in \ddref{DD-taylor-arithmetic-constant-power}.
Set $u = 1/2 z^2 - 2 b z + a^2$. Then $\tc{u}{0} = z_0$, $\tc{u}{1} = -2 b$, $\tc{u}{2} = 1$,
and $\tc{u}{k} = 0$ for $k>2$.
Compute $j = k \Rightarrow k-j = 0$, $j = k-1 \Rightarrow k-j = 1$, and $j = k-2 \Rightarrow k-j = 2$.
A scaled $R_c$ is related to the unscaled $\widetilde R_c$ by $h \widetilde R_c = R_c$. As
a matter of notation, $\widetilde R_c = z_0$. Apply \ddref{DD-taylor-arithmetic-constant-power}
to this problem and simplify to observe the statement of \iref{IM-6T}.

~\newline
\noindent
\begin{minipage}{\textwidth}
\renewcommand*{\arraystretch}{1.5}
\begin{tabular}{| p{\colAwidth} | p{\colBwidth}|}
  \hline
  \rowcolor[gray]{0.9}
  Number& IM\refstepcounter{instnum}\theinstnum \label{IM-6TA}\\
  \hline
  Label& 6TA: Model one primary real pole or logarithmic branch point\\
  \hline
  Theorem& 
  \begin{minipage}[t]{0.8\textwidth} 
    Given real constants $a$, $b$, and $s>0$, real expansion point $z_0 = 0$, and scaling $h$,
    $N > \nuses > \minterms > 0$, and real coefficients $\Setbg{c_n} \subset \Rz$
    of the power series $\sum_{n=0}^{\infty} c_n (z-z_0)^n$, set
    $W(i, 1) = 2 c_{k(i)-1}$,
    $W(i, 2) = 2 (k(i) - 1) c_{k(i)-1}$,
    $W(i, 3) =-2 c_{k(i)-2}$,
    $W(i, 4) =-(k(i) - 2) c_{k(i)-2}$, and
    $b(i) = k(i) c_{k(i)}$
    for $k(i)=N-\nuses, \ldots, N$ and $i=1, \ldots, \nuses$
    to obtain the linear system $W \beta = b$.

    Let $\beta$ solve $\min_{\beta \in \Rz^4} \|\beta\|_2^2$ such that $W \beta = b$.
    Then the least squares best fit of the data $\Setbg{c_n}$ to the
    model problem \eqref{model-complex-pair-pole} is interpreted as
    $\sqrt{\beta(4)} = h/R_c \defeq h/z_0$,
    $\cos \theta = \beta(2) R_c/h$,
    $s_1 = \beta(1)/\beta(2)$, and
    $s_2 = \beta(3)/\beta(4)$.
    For a viable computation, we maintain that 
    $\beta(4) \geq 0$,
    $-1 \leq \cos \theta \leq 1$, and $s_1 \approx s_2$.
  \end{minipage}\\
  \hline
  Description & Use 6TA to compute $R_c$ and $\mu$ and approximate truncation error\\
  \hline
  Sources& The full details are my own contribution. However the ideas are inspired by conversations
  with G. Corliss and N. Nedialkov.\\
  \hline
    Ref.\ By & Final product\\
  \hline
\end{tabular}
\end{minipage}\\

~\newline
We now justify \iref{IM-6TA}.

\subsubsection*{Inside \iref{IM-6TA}}

Given tolerance \tol,
let \ASref{n} and \ASref{real} hold. \ASref{n} provides the parameter $N$ such that, for all
$m \geq n \geq N$, $| \sum_{k=n}^{m} c_k | < \tol$.
Given real constants $a$, $b$, and $s>0$, real expansion point $z_0 = 0$, and scaling $h$,
$N > \nuses > \minterms > 0$, and real coefficients $\Setbg{c_n} \subset \Rz$
of the power series $\sum_{n=0}^{\infty} c_n (z-z_0)^n$, we formulate the optimization problem.

We use optimal least squares fitting to fit the real
coefficients $\Setbg{c_{\nuses}, \ldots, c_N }$ from any given real sequence $\Setbg{c_n}$
to the sequence $\Setbg{w_n}$ of \iref{IM-6T}.

We rearrange the \iref{IM-6T} system
\EQ
{
  \label{eq:6TA}
      k \tc{w}{k} = 2 \tc{w}{k-1} (k +   s - 1) \frac{h}{z_0} \cos \theta
                    -\tc{w}{k-2} (k + 2 s - 2) \left(\frac{h}{z_0}\right)^2,
                   \quad \text{for} \quad k \geq 1.
}
to
\begin{equation}
  \begin{array}{rcl}
    k \tc{w}{k} & = & 2 \tc{w}{k-1} s \frac{h}{z_0} \cos \theta
                     +2 \tc{w}{k-1} (k - 1) \frac{h}{z_0} \cos \theta\\
                &   &-2 \tc{w}{k-2} s \left( \frac{h}{z_0} \right)^2
                     -  \tc{w}{k-2} (k - 2) \left( \frac{h}{z_0} \right)^2,
  \quad \text{for} \quad k \geq 1.
  \end{array}
\end{equation}
Set
$W(i, 1) = 2 c_{k(i)-1}$,
$W(i, 2) = 2 (k(i)- 1) c_{k(i)-1}$,
$W(i, 3) =-2 c_{k(i)-2}$,
$W(i, 4) =-(k(i)- 2) c_{k(i)-2}$,
and
$b(i) = k(i) c_{k(i)}$ for $k(i)=N-\nuses, \ldots, N$ and $i=1, \ldots, \nuses$
to obtain the linear system $W \beta = b$.

At the optimal solution,
$\beta(1) = s h/R_c \cos \theta$,
$\beta(2) = h/R_c \cos \theta$,
$\beta(3) = s (h/R_c)^2$, and
$\beta(4) = (h/R_c)^2$.
If follows that
$\sqrt{\beta(4)} = h/R_c \defeq h/z_0$,
$\cos \theta = \beta(2) R_c/h$,
$s_1 = \beta(1)/\beta(2)$, and
$s_2 = \beta(3)/\beta(4)$.
For a viable computation, we maintain that 
$\beta(4) \geq 0$,
$-1 \leq \cos \theta \leq 1$, and $s_1 \approx s_2$.

Suppose $R_c$, $s_1$, $s_2$, $\cos \theta$, and $\mu$ computed as in \iref{IM-6TA}
have a small fitting error. The quality for small is not yet defined. Then $R_c$ and $\mu$ satisfy
\ASref{approximate} and \ASref{sixterm}. Moreover using
$\tc{w}{k}$ as $c_k$ in \ASref{n}, we have an estimate for the truncation error.

A heuristically motivated top-line analysis produces a conservative estimate
for the radius of convergence $R_c$ from the slope of a linear upper envelope
of a graph of $\log_{10} | c_n |$ versus $n$ \citep{chang1982}. While no proof
is given in this paper, the following argument justifies their claim that
the slope approaches $\log_{10} |z - z_0|/R_c$ as $n \rightarrow \infty$.

\noindent
\begin{minipage}{\textwidth}
\renewcommand*{\arraystretch}{1.5}
\begin{tabular}{| p{\colAwidth} | p{\colBwidth}|}
  \hline
  \rowcolor[gray]{0.9}
  Number& IM\refstepcounter{instnum}\theinstnum \label{IM-rc}\\
  \hline
  Label& \bf Approximating the radius of convergence\\
  \hline
  Inputs & $N$ and approximation points \eqref{approximation-points} \\
  \hline
  Output& $R_c$ or confirmation of divergence.\\
  \hline
  Description& Approximate the radius of the circle of convergence, $R_c \approx 1/10^m$.\\
  \hline
  Sources& The full details are my own contribution. However the ideas are inspired by conversations
  with G. Corliss and N. Nedialkov.\\
  \hline
  Ref.\ By & \iref{IM-order-of-singularity}\\
  \hline
\end{tabular}
\end{minipage}\\

Given $\tol>0$ and any sequence $\Setbg{c_n}$, construct the power series
$\sum_{n=0}^{\infty} c_n (z-z_0)^n$.
Use \ASref{n} to obtain an integer $N$ such that,
for all $m \geq n \geq N$, $| \sum_{k=n}^{m} c_k | < \tol$.
It is no loss in generality to assume $N>30$, else replace $N$ with $30$.
\tref{TM-series-cauchy-condition} says that $\sum_{n=0}^{\infty} c_n$ converges.

We now invoke \ASref{approximate} and extract $15$ elements 
$\Setbg{c_{N-14}, c_{N-13}, \ldots, c_{N}}$ from the sequence.
With $k = i+N-14$, obtain the best linear fit $y(k) = m k + b$ in the $2$-norm to the points
\EQ
{
  \label{eq:approximation-points}
  \Setbg{(N-14, \log_{10} | c_{N-14} |), (N-13, \log_{10} |c_{N-13}|), \ldots, (N, \log_{10} |c_{N}|)},
}
that is, find $m$ and $b$ such that $\sum_{i=0}^{14} |\log_{10} |c_{i+N-14}| - y(i+N-14)|^2$ is minimized.
Because $\sum_{n=0}^{\infty} c_n$ converges, \tref{TM-convergence-of-sequence} says that
$c_n \rightarrow 0$ as $n \rightarrow \infty$. The model parameter $m$ will be negative.

Compute the ratio in the ratio test \tref{TM-ratio-test} with our model. Then 
\EQ
{
  \log_{10} \left| \tfrac{y(k+1)}{y(k)} \right| = \log_{10} | y(k+1) | - \log_{10} | y(k) | = m,
}
which is independent of $k$. By continuity of $\log_{10}$ and the best linear fit function $y$ as well as
$\log_{10} \left| \tfrac{y(k+1)}{y(k)} \right|$ approximates $\log_{10} \left| \tfrac{c_{k+1}}{c_k} \right|$,
we observe that $\log_{10} \left| \tfrac{y(k+1)}{y(k)} \right| \rightarrow m$ is approximately
$\log_{10} \left| \tfrac{c_{k+1}}{c_k} \right| \rightarrow m$. By \tref{TM-comparing-ratio-and-root},
the ratio test \tref{TM-ratio-test} converging implies the root test \tref{TM-root-test} converges, because
they converge to a single limit and the same limit. See description section of \tref{TM-comparing-ratio-and-root}.
It follows from \dref{GD-rc} that the power series converges whenever $m<0$ and diverges whenever $m>0$.
Moreover \dref{GD-rc} says $R_c = 1/10^m$. 

~\newline

\noindent
\begin{minipage}{\textwidth}
\renewcommand*{\arraystretch}{1.5}
\begin{tabular}{| p{\colAwidth} | p{\colBwidth}|}
  \hline
  \rowcolor[gray]{0.9}
  Number& IM\refstepcounter{instnum}\theinstnum \label{IM-order-of-singularity}\\
  \hline
  Label& \bf Compute $\mu$ the order of singularity and underestimate $R_c$\\
  \hline
  Inputs & Tolerance \tol, $N$, and approximation points \eqref{approximation-points} \\
  \hline
  Output& $R_c$ and $\mu$\\
  \hline
  Description&
Start with the series resulting from integration of the given series three times and
fit the coefficients with \iref{IM-rc}. If that graph is linear, meaning the minimizer has norm less
  than \tol, then the slope is accepted and the order of the singularity is 3. If the graph opens upward,
then the series is differentiated term-wise to reduce the second derivative of the graph, and
a new top-line fit is computed. This process is repeated, reducing $\mu$ by 1 each time,
  until the graph opens downward or until
seven term-wise differentiations have been tested. If seven term-wise differentiations have been
tested and each result in turn proves unsatisfactory, then the final estimate for $R_c$ is reduced
by 10 percent for a conservative estimate for $R_c$ and $\mu=-4$ is returned.\\
  \hline
  Sources& \cite{chang1982}\\
  \hline
  Ref.\ By & Final product.\\
  \hline
\end{tabular}
\end{minipage}\\


Let the conditions of Data Definition \ddref{DD-order-of-singularity} hold.
Then the real coefficients $\Setbg{c_n} \subset \Rz$ of the power series
$\sum_{n=0}^{\infty} c_n (z-z_0)^n$ are obtained as a TS solution of an \ode and
consider finding the order $\mu$ of the singularity from the graph
of $\log_{10} | c_n |$ versus $n$.

The order $\mu$ is increased or decreased by term-by-term differentiation or
integration, respectively. The upper envelope of the graph of
$\log_{10} | c_n |$ versus $n$ is concave up for orders $\mu < 1$ which means
the slope underestimates $\log_{10} |z - z_0|/R_c$, and $R_c$ is overestimated.

To estimate $R_c$ and the order $\mu$ form the graph of $\log_{10} | c_n |$ versus $n$,
shift the order of the series by repeated term-wise differentiation or integration. After
each shift, a linear upper envelope is fit with \iref{IM-rc}. The singularity may occur with any order.
However, it is unusual for a solution to a differential equation to have singularities whose 
order lies beyond $| \mu - 1 | \leq 3$ \citep{chang1982}.

~\newline

\subsubsection{Input Data Constraints} \label{sec_DataConstraints}    

Given tolerance \tol, the number of terms of the sequence should be sufficient so that \ASref{n}
  holds. We must know an integer $N$ such that, for all $m \geq n \geq N$,
  $| \sum_{k=n}^{m} c_k | < \tol$. Moreover if this $N<30$, then set $N=30$ for a sufficient
  number of terms to do analysis.

As a second point, the terms in the sequence should not be near overflow/underflow. If this
is the case, then the algorithm will properly scaled the sequence.

\subsubsection{Properties of a Correct Solution} \label{sec_CorrectSolution}

\progname{f} properties of a correct solution are: $R_c$, truncation error, and modelling error
must be positive.

\section{Requirements}\label{sc:requirements}

This section provides the functional requirements, the business tasks that the
software is expected to complete, and the nonfunctional requirements, the
qualities that the software is expected to exhibit.

\subsection{Functional Requirements}

\progname{f} is the implementation of \iref{IM-rc} in the complex case or
\iref{IM-order-of-singularity} in the real case for TS solutions of \ode.

\begin{itemize}
  \item[\rlabel{rISoftware}] Input acquisition via software.
  \item[\rlabel{rIFormat}] Validate input format.
  \item[\rlabel{rIType}] Validate input type.

  \item[\rlabel{rAssumptions}] Inputs should satisfy the assumptions.
  \item[\rlabel{R_Inputs}] Inputs should be scaled to prevent overflow/underflow.
  
  \item[\rlabel{rOSoftware}] Output via software.
  \item[\rlabel{rOFormat}] Output format.
  \item[\rlabel{rOType}] Output type.
  \item[\rlabel{rPARAM}] Parameter acquisition.
  \item[\rlabel{rPFormat}] Parameter format.
  \item[\rlabel{rPType}] Parameter type.
  \item[\rlabel{rPDistribution}] Parameter distribution.
  \item[\rlabel{rPConstraints}] Parameter constraints.
  \item[\rlabel{rPoleRealSolverAlgorithm}] Algorithm to find the distance to the nearest real pole.
  \item[\rlabel{rPoleComplexSolverAlgorithm}] Algorithm to find the distance to the nearest complex conjugate pair of poles.
  \item[\rlabel{rPoleComplicatedAlgorithm}] Algorithm to find the distance to the nearest pole in hard to resolve case.
  \item[\rlabel{rPoleRealSolver}] Find the distance to the nearest real pole.
  \item[\rlabel{rPoleComplexSolver}] Find the distance to the nearest complex conjugate pair of poles.
  \item[\rlabel{rPoleTopLine}] Find the distance to the nearest pole in hard to resolve case.
  \item[\rlabel{rROC}] Pole identification, distinguish a real pole from a complex conjugate pair of poles
    from a complicated situation.
  \item[\rlabel{R_Calculate}] \progname{f} should be developed in \cpp.
\end{itemize}

\subsection{Nonfunctional Requirements}

\noindent \begin{itemize}
  \item[NFR\refstepcounter{nfrnum}\thenfrnum \label{NFR_timing}:]\textbf{Timing:} 
    \progname{f} should execute as fast as the \cite{chang1982} software \rdcon.

    The method developed in this
    project is expected to be independent of system constraints. However
    most TS methods are developed in \cpp or \fortran, the goto languages
    of scientific computing. Certainly a scripting language would not be
    sufficient for large systems.

  \item[NFR\refstepcounter{nfrnum}\thenfrnum \label{NFR_accuracy}:]\textbf{Accuracy:}
    \progname{f} must not overestimate $R_c$.
    
    If $R_c$ is overestimated, then the power-series is a divergence sum on the overestimation.
  
    In \ode solving by TS methods, underestimating $R_c$ is acceptable as an underestimation
    results in a slight increase in computational effort for solving an \ode \ivp.
\end{itemize}

\section{Likely Changes}    

The likely changes for \progname{f} are enumerated in the design documents.

\section{Unlikely Changes} 

The unlikely changes for \progname{f} are enumerated in the design documents.

\section{Traceability Matrices and Graphs}

The purpose of the traceability matrices is to provide easy references on what
has to be additionally modified if a certain component is changed.  Every time a
component is changed, the items in the column of that component that are marked
with an ``X'' may have to be modified as well.  Table~\ref{Table:trace} shows the
dependencies of theoretical models, general definitions, data definitions, and
instance models with each other. Table~\ref{Table:R_trace} shows the
dependencies of instance models, requirements, and data constraints on each
other. Table~\ref{Table:A_trace} shows the dependencies of theoretical models,
general definitions, data definitions, instance models, and likely changes on
the assumptions.

\afterpage{
%\begin{landscape}
\begin{table}[h!]
\centering
\begin{tabular}{|c|c|c|c|c|c|c|}
\hline
	& \aref{as:n}& \aref{as:approximate}& \aref{as:real}& \aref{as:threeterm}& \aref{as:sixterm}& \aref{as:topline} \\
\hline
  \tref{TM-series-cauchy-condition}           &  &  &  & x& x& x\\ \hline
  \tref{TM-convergence-of-sequence}           &  &  &  & x& x& x\\ \hline
  \tref{TM-root-test}                         &  &  &  & x& x& x\\ \hline
  \tref{TM-ratio-test}                        &  &  &  & x& x& x\\ \hline
  \tref{TM-comparing-ratio-and-root}          &  &  &  & x& x& x\\ \hline
  \tref{TM-primary-poles}                     &  &  &  & x& x& x\\ \hline
  \dref{GD-rc}                                &  &  &  & x& x& x\\ \hline
  \ddref{DD-order-of-singularity}             &  &  &  &  &  & x\\ \hline
  \ddref{DD-taylor-arithmetic-constant-power} &  &  &  & x& x&  \\ \hline
  \iref{IM-3T}                                & x& x& x& x&  &  \\ \hline
  \iref{IM-3TA}                               & x& x& x& x&  &  \\ \hline
  \iref{IM-6T}                                & x& x& x&  & x&  \\ \hline
  \iref{IM-6TA}                               & x& x& x&  & x&  \\ \hline
  \iref{IM-rc}                                & x& x& x& x& x& x\\ \hline
  \iref{IM-order-of-singularity}              & x& x& x&  &  & x\\
\hline
\end{tabular}
\caption{Traceability Matrix Showing the Connections\\ Between Assumptions and Other Items}
\label{Table:A_trace}
\end{table}
%\end{landscape}
}

\newgeometry{margin=1cm} % modify this if you need even more space
\begin{landscape}
\begin{table}[h!]
\centering
\begin{tabular}{|c|c|c|c|c|c|c|c|c|c|c|c|c|c|c|c|}
\hline        
	& \tref{TM-series-cauchy-condition}
  & \tref{TM-convergence-of-sequence}
  & \tref{TM-root-test}
  & \tref{TM-ratio-test}
  & \tref{TM-comparing-ratio-and-root} 
  & \tref{TM-primary-poles}
  & \dref{GD-rc}
  & \ddref{DD-order-of-singularity}
  & \ddref{DD-taylor-arithmetic-constant-power}
  & \iref{IM-3T}
  & \iref{IM-3TA}
  & \iref{IM-6T}
  & \iref{IM-6TA}
  & \iref{IM-rc}
  & \iref{IM-order-of-singularity}
  \\
\hline
  \tref{TM-series-cauchy-condition}              &x& & & & & & & & & & & & &x& \\ \hline
  \tref{TM-convergence-of-sequence}              & &x& & & & & & & & & & & &x& \\ \hline
  \tref{TM-root-test}                            & & &x& & & &x& & & & & & &x& \\ \hline
  \tref{TM-ratio-test}                           & & & &x& & & & & & & & & &x& \\ \hline
  \tref{TM-comparing-ratio-and-root}             & & & & &x& & & & & & & & &x& \\ \hline
  \tref{TM-primary-poles}                        & & & & & &x& & & &x&x&x&x&x& \\ \hline
  \dref{GD-rc}                                   & & & & & & &x& & & & & & &x& \\ \hline
  \ddref{DD-order-of-singularity}                & & & & & & & &x& & & & & & &x\\ \hline
  \ddref{DD-taylor-arithmetic-constant-power}    & & & & & & & & &x&x& &x& & & \\ \hline
  \iref{IM-3T}                                   & & & & & & & & & &x&x& & & & \\ \hline
  \iref{IM-3TA}                                  & & & & & & & & & & &x& & & & \\ \hline
  \iref{IM-6T}                                   & & & & & & & & & & & &x&x& & \\ \hline
  \iref{IM-6TA}                                  & & & & & & & & & & & & &x& & \\ \hline
  \iref{IM-rc}                                   & & & & & & & & & & & & & &x&x\\ \hline
  \iref{IM-order-of-singularity}                 & & & & & & & & & & & & & & &x\\
\hline
\end{tabular}
\caption{Traceability Matrix Showing the Connections\\Between Items of Different Sections}
\label{Table:trace}
\end{table}
\end{landscape}
\restoregeometry

\begin{table}[h!]
\centering
\begin{tabular}{|c|c|c|c|c|c|c|}
\hline
	& \iref{IM-3T}
	& \iref{IM-3TA}
	& \iref{IM-6T}
	& \iref{IM-6TA}
	& \iref{IM-rc}
  & \iref{IM-order-of-singularity}
  \\
\hline
\rref{rISoftware}                  &x&x&x&x&x&x\\ \hline
\rref{rIFormat}                    &x&x&x&x&x&x\\ \hline
\rref{rIType}                      &x&x&x&x&x&x\\ \hline
\rref{rAssumptions}                &x&x&x&x&x&x\\ \hline
\rref{R_Inputs}                    &x&x&x&x&x&x\\ \hline
\rref{rOSoftware}                  &x&x&x&x&x&x\\ \hline
\rref{rOFormat}                    &x&x&x&x&x&x\\ \hline
\rref{rOType}                      &x&x&x&x&x&x\\ \hline
\rref{rPARAM}                      &x&x&x&x&x&x\\ \hline
\rref{rPFormat}                    &x&x&x&x&x&x\\ \hline
\rref{rPType}                      &x&x&x&x&x&x\\ \hline
\rref{rPDistribution}              &x&x&x&x&x&x\\ \hline
\rref{rPConstraints}               &x&x&x&x&x&x\\ \hline
\rref{rPoleRealSolverAlgorithm}    &x&x& & & & \\ \hline
\rref{rPoleComplexSolverAlgorithm} & & &x&x& & \\ \hline
\rref{rPoleComplicatedAlgorithm}   & & &x&x& & \\ \hline
\rref{rPoleRealSolver}             &x&x& & & & \\ \hline
\rref{rPoleComplexSolver}          & & &x&x& & \\ \hline
\rref{rPoleTopLine}                & & & & &x&x\\ \hline
\rref{rROC}                        &x&x&x&x&x&x\\ \hline
\rref{R_Calculate}                 &x&x&x&x&x&x\\ \hline
\nfrref{NFR_timing}                &x&x&x&x&x&x\\ \hline
\nfrref{NFR_accuracy}              &x&x&x&x&x&x\\ \hline
  \hline
\end{tabular}
\caption{Traceability Matrix Showing the Connections\\Between Requirements and Instance Models}
\label{Table:R_trace}
\end{table}

\section{Values of Auxiliary Constants}

\progname{f} does not have symbolic parameters at this time.

\newpage

\bibliographystyle {plainnat}
\bibliography {../../refs/References}

\end{document}
