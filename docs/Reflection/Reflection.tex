\documentclass[12pt, titlepage]{article}

\usepackage{amsmath, mathtools}
\usepackage{amsfonts}
\usepackage{amssymb}
\usepackage{booktabs}
\usepackage{tabularx}
\usepackage{xspace}

\usepackage{graphicx}
\usepackage{colortbl}
\usepackage{xr}
\usepackage{longtable}
\usepackage{xfrac}
\usepackage{float}
\usepackage{siunitx}
\usepackage{caption}
\usepackage{geometry}
\usepackage{pdflscape}
\usepackage{afterpage}

\usepackage{fullpage}
\usepackage[round]{natbib}
\usepackage{multirow}
%\usepackage{refcheck}
\usepackage{lipsum}

\usepackage[section]{placeins}
\usepackage{array}

\input{../Comments}
%% Common Parts

\newcommand{\progname}{roc} % PUT YOUR PROGRAM NAME HERE %Every program
                                 % should have a name


\makeatletter
\newcommand*{\addFileDependency}[1]{% argument=file name and extension
  \typeout{(#1)}
  \@addtofilelist{#1}
  \IfFileExists{#1}{}{\typeout{No file #1.}}
}
\makeatother

\newcommand*{\myexternaldocument}[1]{%
    \externaldocument{#1}%
    \addFileDependency{#1.tex}%
    \addFileDependency{#1.aux}%
}

\externaldocument{../SRS/SRS}

\newcommand{\rref}[1]{(R\ref{#1})}
\newcommand{\nfrref}[1]{NFR\ref{#1}}

\begin{document}

\title{Reflection Report on \progname{L}:
Software estimating the radius of convergence of a power series} 
\author{John M Ernsthausen}
\date{\today}
	
\maketitle

\pagenumbering{roman}

\section{Revision History}

\begin{tabularx}{\textwidth}{p{4cm}p{2cm}X}
\toprule {\bf Date} & {\bf Version} & {\bf Notes}\\
\midrule
24 December 2020 & 1.0 & First submission\\
\bottomrule
\end{tabularx}

~~\newpage

\pagenumbering{arabic}

The \progname{f} project set out to estimate the radius of convergence $R_c$ of a given
power series based on the three term analysis for a real pole, six term analysis for
a pair of complex conjugate poles, and top line analysis \cite[pp.~127--128]{chang1982}.
The inputs are a truncated finite sequence of scaled coefficients
and their scaling. The output is $R_c$ and the order of the singularity $\mu$.

\section{Project Overview}

Our objective is correctness, accuracy, and timing. \progname{f} is an idea for an academic
research project. The impression from this software will guide future decisions about
this idea as a viable research project.

\section{Key Accomplishments}

In general, this project went very well. Our research group can be confident that \progname{f}
is achievable. All the achievements were accomplished in this past semester through
CSE 701 which taught scientific computing with \cpp and CSE 741 which taught scientific
computing documentation.

\section{Key Problem Areas}

I don't think anything went wrong. I'm grateful for the opportunities afforded me to
persue this scientific computing software development project.

\section{What Would you Do Differently Next Time}

Nothing. The process just takes a whole lot of time to do it right.

\bibliographystyle{plainnat}

\bibliography{../../refs/References}

\end{document}
