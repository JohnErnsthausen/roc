\documentclass[12pt, titlepage]{article}

\usepackage{amsmath, mathtools}
\usepackage{amsfonts}
\usepackage{amssymb}
\usepackage{booktabs}
\usepackage{tabularx}
\usepackage{xspace}

\usepackage{graphicx}
\usepackage{colortbl}
\usepackage{xr}
\usepackage{longtable}
\usepackage{xfrac}
\usepackage{float}
\usepackage{siunitx}
\usepackage{caption}
\usepackage{geometry}
\usepackage{pdflscape}
\usepackage{afterpage}

\usepackage{fullpage}
\usepackage[round]{natbib}
\usepackage{multirow}
%\usepackage{refcheck}
\usepackage{lipsum}

\usepackage[section]{placeins}
\usepackage{array}

%% Comments

\usepackage{color}

\newif\ifcomments\commentsfalse

\ifcomments
\newcommand{\authornote}[3]{\textcolor{#1}{[#3 ---#2]}}
\newcommand{\todo}[1]{\textcolor{red}{[TODO: #1]}}
\else
\newcommand{\authornote}[3]{}
\newcommand{\todo}[1]{}
\fi

\newcommand{\wss}[1]{\authornote{blue}{SS}{#1}} 
\newcommand{\plt}[1]{\authornote{magenta}{TPLT}{#1}} %For explanation of the template
\newcommand{\an}[1]{\authornote{cyan}{Author}{#1}}
\newcommand{\jme}[1]{\authornote{cyan}{JME}{#1}}

% Common Parts

\usepackage{tabularx}
\usepackage{booktabs}
\usepackage{ifthen}
\usepackage{amsmath,amssymb,xspace}

%PUT YOUR PROGRAM NAME HERE %Every program should have a name
\newcommand{\progname}[1]
{%
  \ifthenelse{\equal{#1}{n}}{{\normalsize ROC\xspace}}{}%
  \ifthenelse{\equal{#1}{L}}{{\Large ROC\xspace}}{}%
  \ifthenelse{\equal{#1}{f}}{{\footnotesize ROC\xspace}}{}%
}
% Abbreviations
\newcommand{\ode}{{\footnotesize ODE}\xspace}
\newcommand{\dae}{{\footnotesize DAE}\xspace}
\newcommand{\ivp}{{\footnotesize IVP}\xspace}
\newcommand{\fadbad}{{\footnotesize FADBAD++}\xspace}
\newcommand{\plainc}{{\footnotesize C}\xspace}
\newcommand{\cpp}{{\footnotesize C++}\xspace}
\newcommand{\adolc}{{\footnotesize ADOL-C}\xspace}
\newcommand{\rdcon}{{\footnotesize DRDCV}\xspace}
\newcommand{\fortran}{{\footnotesize FORTRAN 77}\xspace}
\newcommand{\daets}{{\footnotesize DAETS}\xspace}
\newcommand{\matlab}{{\footnotesize MATLAB}\xspace}
\newcommand{\mathematica}{{\footnotesize MATHEMATICA}\xspace}
\newcommand{\maple}{{\footnotesize MAPLE}\xspace}
% User commands
\newcommand{\Quote}[1]{``{#1}''}
\newcommand{\Ni}{\noindent}
% Environment
\newcommand{\EQ}[1]{\begin{align} {#1} \end{align}}
% Mathematics
\newcommand{\parn}[1]{( {#1} )}
\newcommand{\parbg}[1]{\left(  {#1} \right)}
\newcommand{\pbg}[1]{\bigl(  {#1} \bigr)}
\newcommand{\Setbg}[1]{\bigl\{ {#1} \bigr\}}
\def\Rz{\mathbb{R}}
\def\Cz{\mathbb{C}}
\newcommand{\nliminf}[1]{\liminf\limits_{{#1} \rightarrow \infty}}
\newcommand{\nlimsup}[1]{\limsup\limits_{{#1} \rightarrow \infty}}
% Mathematics: ODE
\newcommand{\iode}{\protect{\makebox{$[t_0,\tend]$}}\xspace}
\newcommand{\lode}{\protect{\makebox{$[t_n,t_{n+1}]$}}\xspace}
\newcommand{\tend}{t_\text{end}}
\newcommand{\tc}[2]{(#1)_{#2}}
\newcommand{\Tp}{T}
\newcommand{\xn}{x_{n}}
\newcommand{\tn}{t_{n}}
% Reference
\renewcommand{\eqref}[1]{(\ref{eq:#1})}
\newcommand{\rrf}[2]{(\ref{eq:#1}--\ref{eq:#2})}
\newcommand{\chref}[1]{\ref{ch:#1}}
\newcommand{\sscref}[1]{\ref{ssc:#1}}
\newcommand{\scref}[1]{Section~\ref{sc:#1}}
\newcommand{\exref}[1]{\ref{ex:#1}}
\newcommand{\rmref}[1]{\ref{rm:#1}}
\newcommand{\apref}[1]{\ref{ap:#1}}
%\newcommand{\tbref}[1]{\ref{tb:#1}}
\newcommand{\dfref}[1]{\ref{df:#1}}
\newcommand{\leref}[1]{\ref{le:#1}}
\newcommand{\fgref}[1]{\ref{fg:#1}}
\newcommand{\coref}[1]{\ref{co:#1}}
\newcommand{\thref}[1]{\ref{th:#1}}
\newcommand{\agref}[1]{\ref{ag:#1}}
\newcommand{\asref}[1]{\ref{as:#1}}
\newcommand{\EQref}[1]{Equation~(\ref{eq:#1})}
\newcommand{\CHref}[1]{Chapter~\ref{ch:#1}}
\newcommand{\SSCref}[1]{Subsection~\ref{ssc:#1}}
\newcommand{\SCref}[1]{Section~\ref{sc:#1}}
\newcommand{\EXref}[1]{Example~\ref{ex:#1}}
\newcommand{\RMref}[1]{Remark~\ref{rm:#1}}
\newcommand{\APref}[1]{Appendix~\ref{ap:#1}}
\newcommand{\TBref}[1]{Table~\ref{tb:#1}}
\newcommand{\DFref}[1]{Definition~\ref{df:#1}}
\newcommand{\LEref}[1]{Lemma~\ref{le:#1}}
\newcommand{\FGref}[1]{Figure~\ref{fg:#1}}
\newcommand{\COref}[1]{Corollary~\ref{co:#1}}
\newcommand{\THref}[1]{Theorem~\ref{th:#1}}
\newcommand{\AGref}[1]{Algorithm~\ref{ag:#1}}
\newcommand{\ASref}[1]{Assumption~\ref{as:#1}}



\makeatletter
\newcommand*{\addFileDependency}[1]{% argument=file name and extension
  \typeout{(#1)}
  \@addtofilelist{#1}
  \IfFileExists{#1}{}{\typeout{No file #1.}}
}
\makeatother

\newcommand*{\myexternaldocument}[1]{%
    \externaldocument{#1}%
    \addFileDependency{#1.tex}%
    \addFileDependency{#1.aux}%
}

\externaldocument{../SRS/SRS}

\newcommand{\rref}[1]{(R\ref{#1})}
\newcommand{\nfrref}[1]{NFR\ref{#1}}

\begin{document}

\title{System Verification and Validation Report for \progname{L}:
Software estimating the radius of convergence of a power series} 
\author{John M Ernsthausen}
\date{\today}
	
\maketitle

\pagenumbering{roman}

\section{Revision History}

\begin{tabularx}{\textwidth}{p{4cm}p{2cm}X}
\toprule {\bf Date} & {\bf Version} & {\bf Notes}\\
\midrule
24 December 2020 & 1.0 & First submission\\
\bottomrule
\end{tabularx}

~~\newpage

\section{Symbols, Abbreviations, and Acronyms}

Symbols, abbreviations, and acronyms applicable to \progname{f} are enumerated
in Section 1 of the Software Requirements Document (SRS) \citep{SRS}.

\newpage

\tableofcontents

%\listoftables %if appropriate

%\listoffigures %if appropriate

\newpage

\pagenumbering{arabic}

In this document, we report the results of executing the verification and validation
plan for \progname{f}.

Our objective for \progname{f} is correctness, accuracy, and timing. \progname{f} is
an idea for an academic research project. The impression from this software will
guide future decisions about this idea as a viable research project.

\subsection{Relevant Documentation}

Relevant documentation is completed. All GitHub issues were fully considered and subsequently closed.

Relevant documentation includes the authors Software Requirements Specification (SRS) \citep{SRS}, the
authors Module Guide (MG) \citep{MG}, and the authors Module Interface Specification (MIS) \citep{MIS}.

\section{Verification and Validation Plan}

Verification and Validation of \progname{f} includes automated testing at the module level,
and the system level. We achieved continuous integration through TravisCI.

\subsection{Verification and Validation Team}

The \progname{f} team includes
author
\href{https://github.com/JohnErnsthausen}{John Ernsthausen},
fellow students
\href{https://github.com/LeilaMousapour}{Leila Mousapour},
\href{https://github.com/salahhessien}{Salah Gamal aly Hessien},
\href{https://github.com/liziscool}{Liz Hofer},
and
\href{https://github.com/XingzhiMac}{Xingzhi Liu} as well as Professors
\href{http://baraksh.com}{Barak Shoshany},
\href{https://github.com/smiths}{Spencer Smith},
\href{https://www.cs.mu.edu/~george/}{George Corliss},
and
\href{http://www.cas.mcmaster.ca/~nedialk}{Ned Nedialkov}.
The author appreciates the helpful comments and superior guidance on this project.

\subsection{SRS Verification Plan}

The SRS document Verification and Validation Plan for \progname{f} was peer-reviewed by
secondary reviewer Salah Gamal aly Hessien and Prof. Spencer Smith the course instructor.

The SRS document was published to \href{https://github.com/JohnErnsthausen/roc}{GitHub}.
Defects were addressed with issues on the GitHub platform.

\subsection{Design Verification Plan}

The Design documents MG and MIS for \progname{f} were peer-reviewed by
domain expert Leila Mousapour and Prof. Spencer Smith the course instructor.

The MG and MIS documents were published to \href{https://github.com/JohnErnsthausen/roc}{GitHub}.
Defects will be addressed with issues on the GitHub platform.

\subsection{Implementation Verification Plan}

Every effort was made to complete every aspect promised in this documentation.

\section{Functional Requirements Evaluation}

\progname{f} does not have system functional requirements at this time.

\section{Nonfunctional Requirements Evaluation}

We considered Timing and Accuracy.

\subsection{Timing}

Gprof output \href{https://github.com/JohnErnsthausen/roc/blob/master/examples\_cpp/gprof\_output.txt}{roc/examples\_cpp/gprof\_output.txt} on the Layne-Watson problem \citep{watson1979} shows that \progname{f} did not
consume a measureable amount of CPU time to compute $R_c$ for this problem.
		
\subsection{Accuracy and Comparison to Existing Implementation}	

We exercised 3TA, 6TA, and TLA on the Layne-Watson problem \citep{watson1979}, computing
each analysis at each timestep.
We discovered that fitting errors and truncation errors could be small (less than 1e-10),
but one or the other of 3TA/6TA would align with TLA. This is not what the
author expected. As a result, the decision logic proposed by \cite{chang1982} is in
question.

Careful research needs to be done to determine the correct indicators for which analysis 3TA/6TA/TLA to apply.
As a result the nonfunctional accuracy requirement could not be fulfilled as I don't know the logic for
this at this time.

The author plans to compare and contrast the $R_c$, $\mu$, fitting error, and truncation error
among the three methods 3TA, 6TA, and TLA. These results will be compared to the \cite{chang1982}
existing implementation.

\section{Unit Testing}

Comparison with closed form solutions proved \progname{f} to be accurate. These comparisons
are among the unit tests. The assertions were chosen to be optimal to the number of digits
reported, meaning that rounding the last digit down will cause the test to fail.

All the unit test pass.

\section{Changes Due to Testing}

I'm not sure what this is.

\section{Automated Testing}

All unit tests are automated.

Valgrind, gcov, and clang-format are automated via my top level Ruby script.
		
\section{Trace to Requirements}

Yes.
		
\section{Trace to Modules}		

Yes.

\section{Code Coverage Metrics}

Visit \href{http://www.johnernsthausen.com/experiences/coverage/}{my website} to review gcov output.

\bibliographystyle{plainnat}

\bibliography{../../refs/References}

\end{document}
